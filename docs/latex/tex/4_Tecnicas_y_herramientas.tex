\capitulo{4}{Técnicas y herramientas}

\section{Metodologías}
\subsection{Scrum}
La metodología \textit{Scrum} es un marco de trabajo de desarrollo ágil de software
iterativo (dividido en líneas de tiempo denominados \textit{sprints}) e incremental, 
que surge como antítesis a la gestión de proyectos predictiva (centrada en la planificación,
presupuestos y los plazos de entrega) \cite{enwiki:scrum2023,book:palacio2021}.

\section{Control de versiones}
El control de versiones se realizará mediante \href{https://git-scm.com/}{Git} puesto que, a 
diferencia de otras herramientas de control de versiones, provee de una administración del
repositorio distribuida en lugar de centralizada, lo que permite a los desarrolladores
tener un clon del repositorio en su equipo local \cite{git:2023}. 

\section{Alojamiento del repositorio}
Para el \textit{hosting} del repositorio se ha barajado emplear: \href{https://github.com/}{GitHub}, 
        \href{https://bitbucket.org/product/es/}{Bitbucket},
        \href{https://about.gitlab.com/}{GitLab}.

Se ha optado por emplear \href{https://github.com/}{GitHub} debido a la facilidad de uso
que ofrece y las múltiples funcionalidades disponibles, sumado a la destreza adquirida
durante el desarrollo del grado con esta plataforma de alojamiento de repositorios.

\section{Gestión del proyecto}
Para la gestión del proyecto se ha considerado emplear \href{https://www.zenhub.com/}{ZenHub},
por su integración completa con GitHub.

Esta herramienta proporciona un \textit{canvas} que permite gestionar las 
tareas (\textit{issues}) en el \textit{sprint} ordenándolas en columnas según sea su 
estado y priorizándolas según se encuentren en esta, además de permitir una 
estimación de su duración, asignar su desarrollo a uno o más
desarrolladores e indicar el \textit{milestone} y el \textit{srpint} al que pertenecen
las tareas.

\section{Comunicación}
La comunicación durante desarrollo del proyecto se realiza vía e-mail, debido a la rapidez,
versatilidad y facilidad de la comunicación, además de la familiaridad de las partes con el 
medio de comunicación.

Para las tutorías telemáticas se escogió emplear 
\href{https://www.microsoft.com/es-es/microsoft-teams/log-in}{Microsoft Teams} por la existencia 
de forma predefinida de una cuenta de usuario de la Suite Office de Microsoft proporcionada
por la Universidad.

\section{Entorno de desarrollo}
Debido a la flexibilidad que aporta Visual Studio Code (editor de texto bajo licencia MIT desarrollado por Microsoft \cite{enwiki:vscode2023})
gracias a el entorno proporcionado de forma nativa y a las extensiones disponibles de Microsoft y terceros, se ha optado
por emplearlo para el desarrollo de los scripts de Python, los Python Notebooks, la documentación del repositorio
en Markdown y la realización de la memoria en LaTeX. 

En cuanto al intérprete Python se ha empleado \href{https://www.python.org/downloads/release/python-3913/}{Python 3.9.13} mediante un entorno virtual que permite instalar
las dependencias sin modificar el entorno base.

En lo referente a la compilación de los ficheros LaTeX, se ha optado por emplear 
MikTex debido a la facilidad de actualización e instalación de dependencias.
Para la integración del compilador de LaTeX con Visual Studio Code se emplea 
\href{https://marketplace.visualstudio.com/items?itemName=James-Yu.latex-workshop}{LaTeX Workshop}, 
puesto que permite la compilación automática al guardar los cambios en los ficheros sin realizar
ninguna configuración explícita.

\section{Creación de diagramas}
Para la creación de diagramas en la memoria y anexos se ha empleado diagrams.net (anteriormente 
denominada draw.io), puesto que se trata de una herramienta gratuita y de código abierto que 
puede emplearse de forma \textit{on-line} \cite{misc:wikipediaDiagrams}. 

Además, permite exportar a diferentes formatos como PNG, JPEG, SVG y PDF.

\section{Librerías}
\subsection{Pandas}
\textit{Pandas} es una herramienta de análisis de datos escrita en Python que ofrece estructuras 
de datos y operaciones para manipular tablas numéricas y series temporales 
\cite{eswiki:pandas2023}.

\subsection{Tensorflow}
\textit{Tensorflow} es una librería de código abierto para aprendizaje automático desarrollada
por Google \cite{eswiki:tensorflow2021}.

En concreto se ha optado por emplear su \textit{API}: \href{https://keras.io/}{\textit{Keras}}, 
debido a su modularidad y extensibilidad \cite{eswiki:keras2022}.

\subsection{Matplotlib}
\textit{Matplotlib} es una librería para generar gráficos bidimensionales en Python a partir de 
diferentes estructuras de datos como listas \cite{eswiki:matplotlib2022}.

\subsection{Json}
El módulo JSON de Python proporciona una \textit{API} (interfaz de programación de 
aplicaciones) para manejar objetos en formato JSON (\textit{JavaScript Object 
Notation}) \cite{python:json2023}.

\subsection{Os}
El módulo OS de Python proporciona una manera portable de emplear la funcionalidad del
sistema operativo \cite{python:os2023}.

\subsection{Statsmodel}
Módulo que proporciona diferentes clases y funciones de distintos modelos estadísticos \cite{misc:perktold2023}.

\section{Restful API}
Para obtener los datos meteorológicos de la zona en la que se sitúa el viñedo y poder
realizar la comparación con las muestras obtenidas en el pluviómetro desplegado se han 
estudiado diferentes opciones:
\begin{itemize}
    \item \href{https://openweathermap.org/}{OpenWeatherMap}
    \item \href{https://www.weatherbit.io/}{Weatherbit}
    \item \href{https://www.aemet.es/es/datos_abiertos/AEMET_OpenData}{AEMET OpenData}
\end{itemize}

Todas las opciones requieren de una petición al servidor mediante HTTPS o HTTP, sin 
embargo, en el primer caso a pesar de que se muestran los datos en una frecuencia 
de una hora, el limitante se encuentra en que solo permite recuperar hasta un año atrás.
En el segundo caso los requerimientos se cumplen al completo, pero tras contactar con el
soporte de la \textit{API} no hay respuestas que permitan el acceso al 
\textit{endpoint} correspondiente.

Finalmente, la última opción proporciona datos históricos sin límite, la contraparte es
que la estación meteorológica consultada se encuentra en Aranda de Duero y se tratan de
datos diarios.
