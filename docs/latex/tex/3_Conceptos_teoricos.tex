\capitulo{3}{Conceptos teóricos}
El desarrollo del proyecto cuenta con diferentes fases desde el pre-procesamiento de los muestreos de los sensores y el pluviómetro,
continuando con la selección del modelo de Red Neuronal Artificial que se empleará en la predicción a corto plazo, así como su 
modelado para ser empleada en este problema concreto y su validación.
En esta sección se presentarán los conceptos teóricos en cada etapa para facilitar su comprensión.

\section{Pre-procesamiento de datos}
En el pre-procesamiento de datos se pretende realizar la integración y limpieza de datos, de forma que disminuyan los posibles problemas
de calidad que puedan surgir en los diferentes sistemas de información.
Como norma general el proceso de integración debe ser realizado durante la fase de recopilación de los datos.
La limpieza permite la detección y corrección de los problemas no resueltos en la fase anterior como los valores anómalos (\textit{Outliers}) o faltantes 
\cite{book:hernandez2004}.  


