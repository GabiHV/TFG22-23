\capitulo{3}{Conceptos teóricos}
El desarrollo del proyecto cuenta con diferentes fases desde el pre-procesamiento de los muestreos de los sensores y el pluviómetro,
continuando con la selección del modelo de Red Neuronal Artificial que se empleará en la predicción a corto plazo, así como su 
modelado para ser empleada en este problema concreto y su validación.

En esta sección se presentarán los conceptos teóricos en cada etapa para facilitar su comprensión.

\section{Pre-procesamiento de datos}
En el pre-procesamiento de datos se pretende realizar la integración y limpieza de datos, de forma que disminuyan los posibles problemas
de calidad que puedan surgir en los diferentes sistemas de información.

Como norma general el \textbf{proceso de integración} debe ser realizado durante la fase de recopilación de los datos.
La limpieza permite la detección y corrección de los problemas no resueltos en la fase anterior como los valores anómalos (\textit{Outliers}) o faltantes 
\cite{book:hernandez2004}.  

Tras la integración de los datos de las diferentes fuentes (v.g. bases de datos), se pueden realizar un resumen de características de atributos, en la que
se mostrarán las características generales de estos como medias, mínimos, máximos y valores posibles. 

En esta tabla que proporciona diferente información podríamos obtener información trascendental para proceso de análisis de datos, sobre todo para atributos categóricos.
En el caso de atributos numéricos un mecanismo visual que es especialmente útil es la gráfica de dispersión, que es la técnica de visualización
de datos que se empleará mayoritariamente durante el desarrollo del proyecto \cite{book:hernandez2004}.

En el conjuto de datos podemos encontrar \textbf{valores faltantes}, perdidos o ausentes que pueden ser reemplazados por diferentes razones. Una de ellas es que
el modelo que empleemos puede no tratar bien estos valores o que utilice un mecanismo de tratamiento que no sea adecuado al contexto.
Un problema asociado a la detección de los valores faltantes es que estos no estén representados como nulos, lo que puede introducir sesgo en el conocimiento
extraido \cite{book:hernandez2004}.

Ante esta situación se puede actuar de diferentes maneras:
\begin{itemize}
    \item Ignorarlos (ciertos algoritmos son tolerantes a los valores faltantes).
    \item Eliminar la columna que contiene valores faltantes.
    \item Filtrar las filas: eliminar las filas afectadas, lo que introduce cierto sesgo en muchas ocasiones.
    \item Reemplazar el valor por otro que preserve la media y la varianza del conjunto de datos en caso de atributos numéricos y la moda en atributos
        catregóricos.
        Una forma de reemplazar los valores faltantes es la imputación de datos perdidos, que consiste en predecirlos a partir de otros ejemplos. Existen también
        algoritmos que se emplean tradicionalmente para este fin.
    \item Segmentar: se segmentan las filas por los valores disponibles y se obtiene un modelo por cada uno de los segmentos y se combinan.
    \item Modificar la política de calidad de datos y esperar a que los faltantes estén disponibles.
\end{itemize}

Además de las situaciones anteriores, es posible que el conjunto de datos cuente con \textbf{valores erróneos} que 
se deben detectar y tratar.
La detección de estos campos puede realizarse de diferentes maneras dependiendo de su formato y origen.

En el caso que nos atañe se deben buscar los valores extremos, que no significa que estos sean erróneos, sino que 
estadísticamente se clasifican como anómalos, aunque representen un estado genuino de la realidad.
Con todo y con eso, estos valores pueden suponer un problema para algunos métodos que se basan en el ajuste de pesos
como las redes neuronales.
En otras ocasiones pueden haber datos erróneos que caen en la normalidad, por lo que no pueden ser detectados \cite{book:hernandez2004}.

La falta de detección de estos valores puede resultar en problemas si posteriormente se normalizan los datos, puesto
que la mayoria de datos estarían en un rango muy pequeño y puede haber poca precisión en algunos modelos ante estas situaciones.

