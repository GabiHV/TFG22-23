\apendice{Plan de Proyecto}

\section{Introducción}
En la sección se introduce la fase de planificación del proyecto, tanto la planificación temporal, como
la viabilidad económica y legal.

\section{Planificación temporal}
La planificación del desarrollo del proyecto se realizó siguiendo la metodología ágil \textit{Scrum}, aunque no en su definición completa, puesto
que al ser una única persona en el desarrollo, no permite realizar todos los procedimientos recogidos dentro de los manuales de gestión de proyectos
con metodologías ágiles \cite{book:palacio2021}, como las reuniones diarias ``daily'' para recoger diferentes cuestiones como cúales son los 
problemas que ralentizan el proceso de desarrollo del proyecto.

Por otro lado, la herramienta software que apoyaba su aplicación (ZenHub \cite{misc:zenhub2023}) dejó de estar disponible de forma abierta, de forma
que no pudo seguirse con el modelo \textit{Canvas} que aplicaba la metodología recogiendo en un tablero con varias columnas, según fuera su estado, las 
tareas por cada uno de los \textit{sprints}.

Aun así, se ha continuado aplicando la filosofía ágil en líneas generales:
\begin{itemize}
    \item Se continuó con el desarrollo iterativo incremental mediante los \textit{sprints}.
    \item La duración media de estas iteraciones fueron de dos semanas al comienzo del proyecto y durante
        la mayor parte de su realización, reduciendo el tiempo al final de este.
    \item En la terminación de los \textit{sprints} se realizaba un incremento, manteniendo reuniones con los 
        tutores para la planificación de la iteración entrante y la revisión de posibles errores en el desarrollo.
    \item Las tareas surgidas de la reunión se creaba, estimaban y priorizaban, en un inicio añadiéndolas al tablero
        \textit{Canvas}.
\end{itemize}

La estimación fue realizada empleando los \textit{story points} disponibles en ZenHub, de forma que estos iban de las 
tareas más sencillas de implementar y rápidas, a las más complejas y que requerían de mayor tiempo
de desarrollo.

\subsection{Sprint 0 (18/01/2023 - 31/01/2023)}
En este \textit{sprint} se presentó el proyecto, indicándose los primeros objetivos, lo que se enmarca en la fase de comprensión
del negocio.
Las primeras tareas se correspondían con la fase de comprensión de datos y a su vez la preparación de los mismos, 
puesto que el conjunto proporcionado contaba con diferentes casuísticas de las que en un comienzo se tenía constancia.

Entre los problemas encontrados previos al análisis en profundidad de los datos se encontraban lecturas incorrectas (valores erróneos) 
debido a problemas de \textit{overflow} de variables en los controladores de cada sensor, de forma que cuando la batería bajaba
de los 3.1V, las lecturas en el conjunto de datos se trataban de errores en su mayoría.

Además, en ciertos periodos de tiempo los sensores habían estado desconectados, por lo que se produjeron lecturas nulas (valores faltantes)
que necesitaban ser tratadas.

Por otro lado, el pluviómetro instalado en las dependencias del viñedo en algunas de las muestras arrojaba valores
inverosímiles debido a la forma en la que se encontraba instalado, de manera que el balancín empleado para medir las diferencias de 
precipitaciones entre instantes se activaba artificialmente por el viento.

Durante el desarrollo de la iteración se comenzó con el tratamiento de los datos nulos, tomando la decisión de eliminar las entradas, pues se
contaba con un número elevado de muestras y realizar la recuperación de estas no parecía viable debido a su naturaleza.
Por otro lado, para los problemas relacionados con el pluviómetro se solicitaron claves \textit{API} para poder acceder a los registros 
de estaciones meteorológicas cercanas a la zona y comprobar las muestras obtenidas.

Por otro lado, se comenzó a realizar las gráficas de los datos para intentar encontrar correlaciones en estos y poder hallar diferentes 
errores en el conjunto.

\subsection{Sprint 1 (31/01/2023 - 14/02/2023)}
En este \textit{sprint} se presentaron los avances de la primera iteración y se acordó la continuación del proceso de comprensión de datos y 
preparación de los mismos.

Los esfuerzos se centraron entonces en modificar las gráficas de visualización para continuar con la inspección visual de los sensores 
y comenzar con las muestras del pluviómetro.
Con estas inspecciones se consigue, de esta forma, encontrar anomalías en ciertos sensores como en el caso del segundo y cuarto.

\subsection{Sprint 2 (14/02/2023 - 28/02/2023)}
En la reunión de la iteración se propusieron diferentes objetivos:
\begin{itemize}
    \item En el sensor 2 eliminar con variación excesiva en temperatura, corrigiendo el ruido presente.
    \item En el sensor 3 se encontró que el salto de humedad no era genuino, por lo que había que realizar un estudio para acordar su viabilidad.
    \item En el sensor 4 intentar realizar una media móvil para arreglar variables como la temperatura al emplear variaciones diarias en lugar de 
        muestreos cada 5 minutos.
\end{itemize}

Durante el desarrollo del \textit{sprint} se continuó con la selección de datos, encontrando que el efecto de la lluvia provocaba ciertos cambios 
bruscos como en el caso del sensor 5, pero estando algunos aparentemente no relacionados con este fenómeno.
Se comenzó, por otro lado con la selección y limpieza mediante una columna adicional de validez en los ficheros correspondientes.

Además, se implementó la detección de \textit{Outliers} mediante el rango intercuartílico, para eliminar el posible ruido en el conjunto de datos
de los sensores.

\subsection{Sprint 3 (28/02/2023 - 14/03/2023)}
En la reunión de la iteración se acordó realizar la recuperación de los datos del sensor 8 hasta la variabilidad excesiva de los valores de las
variables observadas, así como el intento de mejorar la calidad de los conjutnos de otros sensores realizando procesos similares.
Por otro lado, se propuso emplear WeatherBit como \textit{API} meteorológica para eliminar las muestras que más distaran de las lecturas del 
pluviómetro desplegado.

Durante el desarrollo del \textit{sprint} se modificaron los umbrales empleados en la detección y manejo de valores extremos, para eliminar la mayoría
cantidad de ruido posible en todos los sensores.
Por otro lado, se fueron finalizando las selecciones de datos de los sensores, realizando de igual forma una recuperación de valores en sensores cuyas 
variabilidades se debían a factores externos (fueron sacados de sus posiciones originales).

En lo referente del sensor 8 se llegó a la conclusión de que buena parte era irrecuperable debido a las variabilidades aleatorias 
con las que parecía contar.

Así mismo, se comenzó con la limpieza de las muestras del pluviómetro empleando los datos de varias \textit{API}, en primer lugar se trató de emplear
la mencionada anteriormente, estando el \textit{endpoint} ideal bajo licencia de pago.
Probando con OpenWeatherMap se presentó el caso al soporte, consiguiendo ampliar los privilegios de usuario para poder acceder a la utilidad requerida, sin embargo, no
se permitía retrotraerse en más de un año en las lecturas registradas.

Finalmente se decide emplear los servicios de la AEMET, debido principalmente a que permitía obtener los datos de más de hace año atrás, sin embargo, no permitía obtener 
registros por hora, sino por día y la estación meteorológica más cercana se encontraba a unos 15 Kilómetros
del despliegue.

\subsection{Sprint 4 (14/03/2023 - 28/03/2023)}
En la reunión del \textit{sprint} se acuerda comenzar con la fase del modelado a la vista de la aparente correcta selección y limpieza de los datos 
proporcionados.
Se plantea la creación de una matriz de correlación para observar cúales de los atributos seleccionar para la creación del modelo.

Durante el desarrollo de la iteración los esfuerzos se centran en la creación del entorno virtual con las dependencias correspondientes que permitan 
realizar los modelados, además de dejar el fichero de pre-procesamiento comentado con las explciaciones pertinentes.

Por otro lado, se realizan avances sustanciales en la memoria del proyecto añadiendo información de la introducción, objetivos y técnicas y herramientas,
además de realizar una reestructuración de ficheros y una modificación de la forma en la que se realizan las gráficas de los datos.

Se comienza con la creación tentativa del modelo neuronal en busca de conocimiento del dominio, que permita profundizar en los módulos empleados para tal
propósito, realizando regresión simple.
Se investiga sobre las redes neruonales recurrentes, más concretamente sobre la aplicación de modelos como \textit{GRU} y \textit{LSTM} en problemas
similares (predicciones de tiempo atmosférico).

\subsection{Sprint 5 (28/03/2023 - 11/04/2023)}
En la reunión de la iteración se muestran los avances en memoria y en la creación de modelos, por otro lado, se plantea la forma de realizar la regresión
empleando Keras.

Durante el \textit{sprint} se modifica la manera en la que se crean los datos de entrada a los modelos, empleando medias diarias para tal propósito y 
dividiendo en dos conjuntos: entrenamiento y validación. Además, se crean las gráficas de dispersión que permiten comparar visualmente
las predicciones y valores reales.

Por otro lado, se continúa con el desarrollo de la memoria, añadiendo conceptos teóricos sobre el proceso de extracción de conocimiento de bases de datos 
(\textit{KDD}), así como modificando algunos apartados.

\subsection{Sprint 6 (11/04/2023 - 25/04/2023)}
En la planificación se considera entrenar a los modelos neuronales únicamente con valores adecuados, pues hasta ahora pueden existir saltos temporales que
afecten a la regresión, realizando medias temporales cada cierto periodo de tiempo para eliminar los posibles problemas de \textit{offset} de los 
muestreos originales.

En el desarrollo del \textit{sprint}, se modificó la selección en los datos de entrada a los modelos, de forma que se realizaban las medias diarias de los
diferentes atributos y se les aplicaba el filtro de Hodrick-Presscott, un filtro que permite obtener las tendencias de una serie temporal,
de manera que los modelos sean capaces de generalizar de forma más precisa.

Por otro lado, se continuó con la memoria, en concreto con los conceptos teóricos relacionados con la fase de modelado.

\subsection{Sprint 7 (25/04/2023 - 09/05/2023)}
En la reunión de la iteración se revisan los cambios realizados, llegando a la consideración de emplear varios dias previos como entrada a los modelos,
utilizando una ventana deslizante para seleccionar datos, de manera que se eviten los saltos temporales y estudiar la incorporación de una componente temporal 
como entrada en la regresión.

De esta forma, en el desarrollo se aplicó la ventana mencionada para la selección de datos de entrada en el modelo, de modo que se pudieran evitar la situación
anterior.

\subsection{Sprint 8 (09/05/2023 - 16/05/2023)}
En la planificación se acordó intentar rebajar los periodos de tiempo, emplear datos medios por hora en lugar de diarios y probar los hiperparámetros
de las redes, intentando variar las neuronas por capa, el número de estas, el ratio de aprendizaje, etc. para observar el comportamiento de los modelos y
determinar tanto si continúan generalizando como si es necesario realizar cambios en el proceso de conocimiento de datos y preparación de estos.

De esta manera, se redujo la frecuencia de las medias de los datos, realizando su integración en un único fichero, para poder
entrenar los modelos empleando un único conjunto. Esto obligó a cambiar la forma en la que se graficaban los resultados, mostrando las gráficas de 
dispersión por sensor y atributo.

Por otro lado, se dividió el conjunto de datos en tres subconjuntos diferentes: un conjunto para entrenamiento, otro para validación y el restante para test, 
permitiendo la parametrización en la división de estos.

\subsection{Sprint 9 (16/05/2023 - 23/05/2023)}
En la reunión de la iteración se acordó tratar de predecir tiempos más largos en lugar de un único instante como realizaban los modelos hasta el momento;
previsiblemente el error sería mayor, por lo que se mencionó permitir parametrizar este valor.

Ante el objetivo principal planteado hubo que realizar diferentes modificaciones en la selección de los datos de salida y los parámetros, penalizando el rendimiento
general de los modelos.

Se avanzó en la memoria en los conceptos teóricos y se realizaron algunas correcciones.

\subsection{Sprint 10 (23/05/2023 - 30/05/2023)}
En la planificación se indicó la continuación en los esfuerzos de realización de memoria y anexos.
De esta manera, este \textit{sprint} se centró en la finalización de los mismos.

\subsection{Sprint 11 (30/05/2023 - 06/06/2023)}
En este \textit{sprint} se acordó realizar más pruebas cambiando el número de predicciones en las salidas de los modelos,
empleando para ello herramientas como Google Colab, de manera que pudieran agilizarse los procesos para obtener los resultados.

Se propuso, por otro lado, la utilización nativa de librerías de \textit{IA} empleando \textit{CUDA}.

\section{Estudio de viabilidad}
En esta sección se desglosarán la viabilidad económica y legal del proyecto; en cuanto a la primera indicará los costes derivados del desarrollo en un enotorno real.
En lo referente la segunda, se presentarán las licencias empleadas en el proyecto y su implicación con librerías de terceros.

\subsection{Viabilidad económica}
En términos de viabilidad económoca es necesario hacer una diferenciación entre los costes y los beneficios que conlleva realizar el proyecto.

\subsubsection{Costes}
Los costes que pueden surgir del proyecto en un entorno empresarial pueden desglosarse en los siguientes:

\textbf{Costes de personal:}

El desarrollo se ha realizado con un única persona empleada a tiempo completo en aproximadamente 4 meses, de forma que se consideran los siguientes 
costes:
\tablaSmallSinColores{Costes de Hardware}{l l}{Coste Hardware}{\textbf{Concepto} & \textbf{Coste} \\}
{Salario mensual neto & 1.080€ \\
Retención I.R.P.F(12\%) & 185,46€ \\
Seguridad Social(31,1\%) & 480,64€ \\
\hline
Salario mensual bruto & 1.545€ \\
\hline
\textbf{Total 4 meses} & 6.181,92€ \\
}

El porcentaje mensual de aporte a la Segurida Social se calcula como el 0,2\% al Fondo de Garantía Salarial (FOGASA), más el 0,6\%, el 6,7\% de
la prestación por desempleo y el 23,6\% de contingencias comunes \cite{misc:ss2023}.

\textbf{Costes de hardware:}

El hardware empleado tiene un tiempo de amortización aproximado de 4 años.
\tablaSmallSinColores{Costes de Hardware}{l l l}{Coste Hardware}{\textbf{Concepto} & \textbf{Coste} & \textbf{Coste amortizado} \\}
{Ordenador portátil & 900€ & 45€ \\
\hline
\textbf{Total} & 900€ & 45€ \\
}

\textbf{Costes de software:}

En cuanto a los costes de software, hay ciertos programas o herramientas empleadas durante el desarrollo que requieren de licencia de pago,
estos contarán con un tiempo de amortización estimado de 2 años.
\tablaSmallSinColores{Costes de Software}{l l l}{Coste Software}{\textbf{Concepto} & \textbf{Coste} & \textbf{Coste amortizado} \\}
{Windows 10 Home & 148€ & 18,5€ \\
\hline
\textbf{Total} & 148€ & 18,5€ \\
}

\textbf{Costes varios:}

Al igual que en otras situaciones empresariales, en los desarrollos de software surgen tanto costes inesperados, como fijos.
Se reflejarán aquellos costes variados que aparentemente se tratarían de costes fijos.
\tablaSmallSinColores{Costes varios}{l l}{Coste varios}{\textbf{Concepto} & \textbf{Coste} \\}
{Memoria y anexos & 50€ \\
Alquiler de espacio de trabajo & 592€ \\
Internet & 120€ \\
\hline
\textbf{Total} & 762€ \\
}

\textbf{Costes totales:}

Los costes del desarrollo del proyecto son los siguientes:
\tablaSmallSinColores{Costes totales}{l l}{Coste totales}{\textbf{Concepto} & \textbf{Coste} \\}
{Personal & 6.181,92€ \\
Hardware & 900€ \\
Software & 148€ \\
Varios & 762€ \\
\hline
\textbf{Total} & 7.991,92€ \\
}

\subsubsection{Beneficios}
En cuanto a los beneficios generados, en el \textit{scope} del proyecto no había cabida a un despliegue en los meses de desarrollo de los 
modelos, por lo que a corto plazo no habría ningún beneficio aparente.

Sin embargo, la forma de obtener los beneficios entraría dentro de una segunda fase, al emplear los resultados
en algún producto o servicio posterior.

\subsection{Viabilidad legal}
En esta sección se desglosarán los temas relacionados con las licencias de uso de los productos software.

En primer lugar se analizará la licencia más conveniente en el desarrollo del proyecto, teniendo en cuenta las dependencias empleadas y 
sus respectivas licencias, puesto que la selección estará limitada por las restricciones de estas principalmente.

\tablaSmallSinColores{Licencias de las dependencias utilizadas}{m{6em} m{5em} m{12em} m{5em}}{Licencias dependencias}{\textbf{Dependencia} & \textbf{Versión} & \textbf{Descripción} & \textbf{Licencia}\\}
{TensorFlow & 2.11.0 & Biblioteca de aprendizaje automático & Apache v2.0 \\
Pandas & 1.5.3 & Biblioteca especializada en manipulación y análisis de datos & BSD \\
Matplotlib & 2.11.0 & Biblioteca para generación de gráficos & BSD \\
IPython Kernel for Jupyter & 6.21.3 & Biblioteca para manipulación de Python Notebooks & BSD \\
Statsmodel & 0.13.5 & Biblioteca con modelos y funciones estadísticas & BSD \\
}

De esta forma, se debe seleccionar una licencia compatible con Apache v2.0 y BSD, siendo la primera la más restrictiva de las licencias, que requiere la conservación del aviso de derecho
de autor y un descargo de responsabilidad, sin embargo, no es \textit{copyleft} \cite{misc:apache2}.

Por su versatilidad se escoge una licencia MIT, que se trata de una licencia permisiva de software libre, imponiendo muy poocas restricciones y otorgando una
buena compatibilidad con otras licencias como la mencionada anteriormente.
