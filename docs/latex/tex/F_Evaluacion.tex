\apendice{Evaluación}

\section{Introducción}
En esta fase se estima el rendimiento de los diferentes modelos, y, en su caso, 
se reconsideran los objetivos del proyecto, de manera que si estos son poco efectivos,
se vuelve a la primera fase.

En todos los modelos se ha empleado validación cruzada externa y, además, se realizarán
diferentes ejecuciones con diversos parámetros para conocer el rendimiento de estos, a 
la par que poder eliminar en cierta medida el efecto del azar en los modelos 
neuronales resultantes.

Para poder validarlos se realizarán pruebas variando el números de capas
ocultas, así como el número de celdas en cada una de estas y el ratio de aprendizaje.
De esta manera, se realizarán pruebas con 1, 2 y 3 capas ocultas en cada uno de los modelos, 
así como 32, 16 y 8 celdas/neuronas y 0.1, 0.01 y 0.001 de ratio de aprendizaje. 

Por otro lado, se ha establecido el conjunto de datos de entrenamiento al 75\% de los disponibles
por sensor, 15\% para el conjunto de validación siendo el restante para test.

Con respecto al número de muestras de entrada se emplerarán 6 horas previas para realizar
una predicción a una hora vista, empleando un tamaño de bloque de 256 ejemplos (los modelos 
no se entranan con todos los datos en todas las iteraciones, sino que se emplean un subconjunto).

Por otro lado, se establece un planificador de la tasa de aprendizaje, de forma
que en la época 50 disminuye a razón de \(lr * e^{-0.1}\), para de esta manera,
centrar los esfuerzos en la región explorada.

Para cada una de las pruebas se realizarán 5 ejecuciones diferentes con 100 iteraciones cada una, 
tomando como medida comparativa el error cuadrático medio de validación.

\section{Resultados de GRU}
En la Tabla \ref{tabla:Evaluación GRU} se presentan los errores cuadráticos medios
que se obtendrán por término medio con el modelo. 

\tablaSmallSinColores{Evaluación GRU}{l l l l}{Evaluación GRU}{\textbf{Capas} & \textbf{Tamaño capa} & \textbf{Learning rate} & \textbf{MSE GRU} \\}
{
    1 & 8 & 0.001 & 1.72183E-05 \\
    1 & 8 & 0.01 & 3.37952E-06\\
    1 & 8 & 0.1 & 0.00551\\
    1 & 16 & 0.001 & 9.1907E-06\\
    1 & 16 & 0.01 & 9.55771E-07\\
    1 & 16 & 0.1 & 0.00474\\
    1 & 32 & 0.001 & 4.50204E-06\\
    \textbf{1} & \textbf{32} & \textbf{0.01} & \textbf{5.96471E-07}\\
    1 & 32 & 0.1 & 0.00297\\

    % 2 capas ocultas
    2 & 8 & 0.001 & 1.56769E-05\\
    2 & 8 & 0.01 & 1.65277E-06\\
    2 & 8 & 0.1 & 0.00965\\
    2 & 16 & 0.001 & 5.07987E-06\\
    \textbf{2} & \textbf{16} & \textbf{0.01} & \textbf{5.66822E-07}\\
    2 & 16 & 0.1 & 0.01148\\
    2 & 32 & 0.001 & 2.18717E-06\\
    \textbf{2} & \textbf{32} & \textbf{0.01} & \textbf{2.69205E-07}\\
    2 & 32 & 0.1 & 0.01727\\

    % 3 capas ocultas
    3 & 8 & 0.001 & 1.61179E-05\\
    3 & 8 & 0.01 & 1.42658E-06\\
    3 & 8 & 0.1 & 0.00706\\
    3 & 16 & 0.001 & 5.18860E-06\\
    \textbf{3} & \textbf{16} & \textbf{0.01} & \textbf{4.66034E-07}\\
    3 & 16 & 0.1 & 0.02418\\
    3 & 32 & 0.001 & 2.36664E-06\\
    3 & 32 & 0.01 & 8.90246E-07\\
    3 & 32 & 0.1 & 0.00976\\
}

Como puede observarse, los mejores resultados en términos generales les obtendremos 
con un \textit{learning rate} de 0.01.

\section{Resultados de LSTM}

\tablaSmallSinColores{Evaluación LSTM}{l l l l}{Evaluación LSTM}{\textbf{Capas} & \textbf{Tamaño capa} & \textbf{Learning rate} & \textbf{MSE LSTM} \\}
{
    1 & 8 & 0.001 & \\
    1 & 8 & 0.01 & \\
    1 & 8 & 0.1 & \\
    1 & 16 & 0.001 & \\
    1 & 16 & 0.01 & \\
    1 & 16 & 0.1 & \\
    1 & 32 & 0.001 & \\
    1 & 32 & 0.01 & \\
    1 & 32 & 0.1 & \\

    % 2 capas ocultas
    2 & 8 & 0.001 & \\
    2 & 8 & 0.01 & \\
    2 & 8 & 0.1 & \\
    2 & 16 & 0.001 & \\
    2 & 16 & 0.01 & \\
    2 & 16 & 0.1 & \\
    2 & 32 & 0.001 & \\
    2 & 32 & 0.01 & \\
    2 & 32 & 0.1 & \\

    % 3 capas ocultas
    3 & 8 & 0.001 & \\
    3 & 8 & 0.01 & \\
    3 & 8 & 0.1 & \\
    3 & 16 & 0.001 & \\
    3 & 16 & 0.01 & \\
    3 & 16 & 0.1 & \\
    3 & 32 & 0.001 & \\
    3 & 32 & 0.01 & \\
    3 & 32 & 0.1 & \\
}

\section{Resultados de MLP}

\tablaSmallSinColores{Evaluación MLP}{l l l l}{Evaluación MLP}{\textbf{Capas} & \textbf{Tamaño capa} & \textbf{Learning rate} & \textbf{MSE MLP} \\}
{
    1 & 8 & 0.001 & \\
    1 & 8 & 0.01 & \\
    1 & 8 & 0.1 & \\
    1 & 16 & 0.001 & \\
    1 & 16 & 0.01 & \\
    1 & 16 & 0.1 & \\
    1 & 32 & 0.001 & \\
    1 & 32 & 0.01 & \\
    1 & 32 & 0.1 & \\

    % 2 capas ocultas
    2 & 8 & 0.001 & \\
    2 & 8 & 0.01 & \\
    2 & 8 & 0.1 & \\
    2 & 16 & 0.001 & \\
    2 & 16 & 0.01 & \\
    2 & 16 & 0.1 & \\
    2 & 32 & 0.001 & \\
    2 & 32 & 0.01 & \\
    2 & 32 & 0.1 & \\

    % 3 capas ocultas
    3 & 8 & 0.001 & \\
    3 & 8 & 0.01 & \\
    3 & 8 & 0.1 & \\
    3 & 16 & 0.001 & \\
    3 & 16 & 0.01 & \\
    3 & 16 & 0.1 & \\
    3 & 32 & 0.001 & \\
    3 & 32 & 0.01 & \\
    3 & 32 & 0.1 & \\
}