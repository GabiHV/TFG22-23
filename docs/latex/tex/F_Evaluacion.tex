\apendice{Evaluación}

\section{Introducción}
En esta fase se estima el rendimiento de los diferentes modelos, y, en su caso, 
se reconsideran los objetivos del proyecto, de manera que, si estos son poco efectivos,
se vuelve a la primera fase.

En todas las redes neuronales se ha empleado validación cruzada externa y, además, se realizarán
diferentes ejecuciones con diversos parámetros para conocer el rendimiento de estas, a 
la par que poder eliminar en cierta medida el efecto del azar en los modelos 
resultantes.

Para validarlos se realizarán pruebas variando el número de capas
ocultas, así como el de celdas/neuronas en cada una de estas y el ratio de aprendizaje.
De esta manera, se realizarán combinaciones con 1, 2 y 3 capas en cada uno de los modelos, 
así como 32, 16 y 8 celdas/neuronas y 0.1, 0.01 y 0.001 de ratio de aprendizaje. 

Por otro lado, se ha dividido el conjunto de datos original en un conjunto de entrenamiento del 75\% de los disponibles
por sensor, 15\% para validación siendo el restante de test.

Con respecto al número de muestras de entrada se emplearán 6 horas previas para realizar
una predicción, empleando un tamaño de bloque de 256 ejemplos (los modelos 
no se entrenan con todos los datos en todas las iteraciones, sino que se emplean un subconjunto).
En la salida, por otro lado, se realizarán predicciones de 1, 2 y 3 horas (marcadas como MSE 1, MSE 2 y MSE 3 en las
respectivas tablas).

Además, se emplea un planificador de la tasa de aprendizaje, de forma
que en la época 50 disminuye multiplicando \(lr * e^{-0.1}\), para, de esta manera,
centrar los esfuerzos en la región explorada.

Para cada una de las pruebas se realizarán 5 ejecuciones diferentes con 100 iteraciones, 
tomando como medida comparativa el error cuadrático medio de validación, lo que supone 405 ejecuciones por modelo, es decir, 1215 ejecuciones en total.

\newpage

\section{Resultados de GRU}
En la Tabla \ref{tabla:Evaluación GRU} se presentan los errores cuadráticos medios
que se obtendrán para cada una de las pruebas. En esta se puede observar los resultados obtenidos con 1, 2 y 3 predicciones 
como salida de los modelos. 

\tablaSmallSinColores{Evaluación GRU}{l l l l l l}{Evaluación GRU}{\textbf{Capas} & \textbf{Tamaño capa} & \textbf{LR} & \textbf{MSE 1} & \textbf{MSE 2} & \textbf{MSE 3}\\}
{
    1 & 8 & 0.001 & 1.72183E-05 & 3.98320E-05 & 5.84809E-05 \\
    1 & 8 & 0.01 & 3.37952E-06 & 7.99268E-06 & 1.41913E-05\\
    1 & 8 & 0.1 & 0.00551 & 0.00127 & 0.00188 \\
    1 & 16 & 0.001 & 9.1907E-06 & 1.73932E-05 & 3.24371E-05 \\
    1 & 16 & 0.01 & 9.55771E-07 & 2.96765E-06 & 7.32603E-06 \\
    1 & 16 & 0.1 & 0.00474 & 0.00718 & 0.00138 \\
    1 & 32 & 0.001 & 4.50204E-06 & 1.22795E-05 & 2.51439E-05 \\
    1 & 32 & 0.01 & \textbf{5.96471E-07} & 1.69264E-06 & 4.50310E-06 \\
    1 & 32 & 0.1 & 0.00297 & 0.00158 & 0.00175 \\

    % 2 capas ocultas
    2 & 8 & 0.001 & 1.56769E-05 & 2.90745E-05 & 4.93188E-05 \\
    2 & 8 & 0.01 & 1.65277E-06 & 6.30449E-06 & 1.14688E-05 \\
    2 & 8 & 0.1 & 0.00965 & 0.00972 & 0.00026 \\
    2 & 16 & 0.001 & 5.07987E-06 & 1.33184E-05 & 2.59817E-05 \\
    2 & 16 & 0.01 & \textbf{5.66822E-07} & \textbf{1.45485E-06} & \textbf{3.49998E-06} \\
    2 & 16 & 0.1 & 0.01148 & 0.01579 & 0.00363 \\
    2 & 32 & 0.001 & 2.18717E-06 & 7.23831E-06 & 1.42593E-05 \\
    2 & 32 & 0.01 & \textbf{2.69205E-07} & \textbf{9.00049E-07} & \textbf{4.04636E-06} \\
    2 & 32 & 0.1 & 0.01727 & 0.00611 & 0.00797 \\

    % 3 capas ocultas
    3 & 8 & 0.001 & 1.61179E-05 & 2.57331E-05 & 4.20175E-05 \\
    3 & 8 & 0.01 & 1.42658E-06 & 5.39812E-06 & 9.51720E-06 \\
    3 & 8 & 0.1 & 0.00706 & 0.00029 & 0.00242 \\
    3 & 16 & 0.001 & 5.18860E-06 & 1.23398E-05 & 2.02378E-05 \\
    3 & 16 & 0.01 & \textbf{4.66034E-07} & \textbf{1.29760E-06} & \textbf{3.49093E-06}\\
    3 & 16 & 0.1 & 0.02418 & 0.01457 & 0.01497\\
    3 & 32 & 0.001 & 2.36664E-06 & 5.75822E-06 & 1.13415E-05\\
    3 & 32 & 0.01 & 8.90246E-07 & \textbf{9.57532E-07} & \textbf{2.41350E-06}\\
    3 & 32 & 0.1 & 0.00976 & 0.01064 & 0.01097\\
}

Como puede observarse, las mayores eficiencias en términos generales se obtienen 
con un \textit{learning rate} de 0.01, destacando los modelos de 2 capas con 
32 y 16 celdas de memoria respectivamente.

Cabe destacar, además, que, como se podía intuir inicialmente, los errores aumentan generalmente
a medida que realizamos más predicciones como salida.

A continuación se presentan las comparativas de los errores cuadráticos medios promedio para un número
diferente de predicciones (1, 2 y 3 predicciones) del modelo estudiado y un parámetro diferente en cada
ocasión.

En el caso del número de capas, como puede observarse en la Figura \ref{fig:Capas_GRU.png}, en promedio
se obtendrán modelos con menor error generalmente en sistemas con 3 predicciones a la salida, siendo los 
modelos de 1 capa los que mayor precisión han presentado en las pruebas.
\imagen{Capas_GRU.png}{Comparativa de número de capas para GRU}{0.9}

En lo que se refiere a las celdas de memoria, por término medio generalmente se obtendrán modelos más
precisos con estructuras que realizan 3 predicciones por entrada, siendo la configuración de 8 unidades 
la que proporciona modelos con errores menores en promedio.
\imagen{Unidades_GRU.png}{Comparativa de número de unidades para GRU}{0.9}

Como puede apreciarse en la Figura \ref{fig:Lr_GRU.png}, con 0.1 como ratio de aprendizaje se obtienen los
resultados con mayor error en promedio.
\imagen{Lr_GRU.png}{Comparativa de ratio de aprendizaje para GRU}{0.9}

En la Figura \ref{fig:Lr_GRU_ampl.png} puede observarse la diferencia del resto de los valores del 
parámetro. En este caso, se obtienen mejores resultados por término medio con un ratio de aprendizaje de
0.01, siendo los modelos de 1 predicción por entrada los que generalmente obtienen errores menores.
\imagen{Lr_GRU_ampl.png}{Comparativa de ratio de aprendizaje para GRU (0.01 y 0.001)}{0.9}

\newpage

\section{Resultados de LSTM}
Al igual que en la situación anterior, se realizarán pruebas con el mismo número de predicciones
por cada entrada, obteniéndose los resultados reflejados en la Tabla \ref{tabla:Evaluación LSTM}.

\tablaSmallSinColores{Evaluación LSTM}{l l l l l l}{Evaluación LSTM}{\textbf{Capas} & \textbf{Tamaño capa} & \textbf{LR} & \textbf{MSE 1} & \textbf{MSE 2} & \textbf{MSE 3}\\}
{
    1 & 8 & 0.001 & 2.12520E-05 & 3.03812E-05 & 5.60944E-05 \\
    1 & 8 & 0.01 & 3.4556E-06 & 1.05315E-05 & 1.68195E-05 \\
    1 & 8 & 0.1 & 3.9049E-06 & 2.55932E-05 & 2.56167E-05 \\
    1 & 16 & 0.001 & 1.18867E-05 & 1.98377E-05 & 4.04516E-05 \\
    1 & 16 & 0.01 & 1.31837E-06 & 3.46421E-06 & 7.71354E-06 \\
    1 & 16 & 0.1 & 3.14115E-06 & 7.24853E-06 & 7.55877E-05 \\
    1 & 32 & 0.001 & 5.84463E-06 & 1.46677E-05 & 2.83427E-05 \\
    1 & 32 & 0.01 & \textbf{6.62421E-07} & \textbf{1.96414E-06} & 5.22169E-06 \\
    1 & 32 & 0.1 & \textbf{9.51891E-07} & 0.00011 & 1.05361E-05 \\

    % 2 capas ocultas
    2 & 8 & 0.001 & 3.29916E-05 & 4.19003E-05 & 6.14671E-05 \\
    2 & 8 & 0.01 & 4.94574E-06 & 8.51327E-06 & 1.33202E-05 \\
    2 & 8 & 0.1 & 6.85972E-06 & 8.04903E-06 & 1.58783E-05 \\
    2 & 16 & 0.001 & 1.33461E-05 & 2.75447E-05 & 4.1725E-05 \\
    2 & 16 & 0.01 & 1.23340E-06 & \textbf{2.38821E-06} & \textbf{4.63577E-06} \\
    2 & 16 & 0.1 & 6.65626E-06 & 2.42052E-05 & 1.43046E-05 \\
    2 & 32 & 0.001 & 5.63944E-06 & 1.15909E-05 & 2.34114E-05 \\
    2 & 32 & 0.01 & \textbf{4.56945E-07} & \textbf{1.43223E-06} & \textbf{2.78594E-06}\\
    2 & 32 & 0.1 & 4.20038E-06 & 1.00814E-05 & 7.57666E-05\\

    % 3 capas ocultas
    3 & 8 & 0.001 & 4.58400E-06 & 5.81849E-05 & 0.00012 \\
    3 & 8 & 0.01 & 5.90159E-06 & 9.07351E-06 & 1.48364E-05 \\
    3 & 8 & 0.1 & 0.00031 & 0.00486 & 1.73157E-05 \\
    3 & 16 & 0.001 & 2.22344E-05 & 3.47894E-05 & 5.07083E-05 \\
    3 & 16 & 0.01 & 1.17654E-06 & 2.41703E-06 & \textbf{4.06992E-06} \\
    3 & 16 & 0.1 & 0.00487 & 1.02880E-05 & 0.00490 \\
    3 & 32 & 0.001 & 8.16043E-06 & 1.64882E-05 & 2.57956E-05 \\
    3 & 32 & 0.01 & \textbf{6.54021E-07} & \textbf{1.19201E-06} & \textbf{2.58606E-06} \\
    3 & 32 & 0.1 & 0.00491 & 0.01459 & 0.00986 \\
}

De forma similar que en \textit{GRU}, en \textit{LSMT} se generan mejores resultados con un 
\textit{learning rate} de 0.01, destacando, de igual forma, los modelos con 2 capas, además
de los de 3 capas y 32 celdas de memoria.
Al igual que antes, los errores se incrementan, generalmente, con el aumento del número de predicciones.

A continuación, del mismo modo que en la sección anterior, se presentan las comparativas de los errores 
cuadráticos medios promedio para un número diferente de predicciones (1, 2 y 3 predicciones) del modelo estudiado y un parámetro diferente en cada ocasión.

En la Figura \ref{fig:Capas_LSTM.png} se observa que las configuraciones de 2 capas registran los 
errores más bajos en promedio que el resto de valores del parámetro, siendo los modelos de 1 predicción
futura los que generalmente obtendrán mejores resultados.
\imagen{Capas_LSTM.png}{Comparativa de número de capas para LSTM}{0.9}

En el caso del número de celdas de memoria, se obtendrán modelos más precisos por término medio con 8 
unidades por capa (a excepción de los sistemas con 2 predicciones por entrada que contarán con mejor 
desempeño en redes con 16 unidades). 
\imagen{Unidades_LSTM.png}{Comparativa de número de unidades para LSTM}{0.9}

En lo referente al ratio de aprendizaje, los valores del parámetro para los que se obtienen modelos
con menor precisión por término medio será 0.1, seguido de 0.001.
\imagen{Lr_LSTM.png}{Comparativa de ratio de aprendizaje para LSTM}{0.9}

En la Figura \ref{fig:Lr_LSTM_ampl.png} puede observarse que para 0.01 de ratio de aprendizaje se obtienen
los resultados con mayor precisión por término con 1 y 2 predicciones por entrada. Siendo, de esta forma,
la configuración de 1 predicción la que menores errores arroje en promedio en comparación con el resto
de parámetros.
\imagen{Lr_LSTM_ampl.png}{Comparativa de ratio de aprendizaje para LSTM (0.01)}{0.9}

\newpage

\section{Resultados de MLP}
En la Tabla \ref{tabla:Evaluación MLP} se proporcionan los diferentes resultados obtenidos con \textit{MLP}.

\tablaSmallSinColores{Evaluación MLP}{l l l l l l}{Evaluación MLP}{\textbf{Capas} & \textbf{Tamaño capa} & \textbf{LR} & \textbf{MSE 1} & \textbf{MSE 2} & \textbf{MSE 3} \\}
{
    1 & 8 & 0.001 & 0.00076 & 0.00017 & 0.00089 \\
    1 & 8 & 0.01 & 0.01218 & 0.00313 & 0.00229 \\
    1 & 8 & 0.1 & 0.02064 & 0.02415 & 0.02420 \\
    1 & 16 & 0.001 & 6.21736E-05 & 7.85435E-05 & 6.44446E-05 \\
    1 & 16 & 0.01 & 0.00059 & 0.00119 & 8.35920E-05 \\
    1 & 16 & 0.1 & 0.02056 & 0.02415 & 0.02420 \\
    1 & 32 & 0.001 & 1.33441E-05 & \textbf{2.31054E-05} & 4.27180E-05 \\
    1 & 32 & 0.01 & \textbf{5.57840E-06} & \textbf{1.32064E-05} & \textbf{2.49582E-05} \\
    1 & 32 & 0.1 & 0.02415E-05 & 0.02062 & 0.02420 \\

    % 2 capas ocultas
    2 & 8 & 0.001 & 0.00094 & 0.00043 & 0.00085 \\
    2 & 8 & 0.01 & 0.00673 & 0.00332 & 0.00327 \\
    2 & 8 & 0.1 & 0.02428 & 0.02430 & 0.02063 \\
    2 & 16 & 0.001 & 1.96797E-05 & 2.97114E-05 & 0.00013 \\
    2 & 16 & 0.01 & 0.00042 & 0.00040 & 0.00089 \\
    2 & 16 & 0.1 & 0.01624 & 0.02422 & 0.01343 \\
    2 & 32 & 0.001 & 8.32571E-06 & \textbf{2.15010E-05} & 3.63325E-05 \\
    2 & 32 & 0.01 & \textbf{5.56436E-06} & 5.32363E-05 & \textbf{2.65348E-05} \\
    2 & 32 & 0.1 & 0.02032 & 0.02060 & 0.02043 \\

    % 3 capas ocultas
    3 & 8 & 0.001 & 0.00152 & 0.00283 & 0.00195 \\
    3 & 8 & 0.01 & 0.00168 & 0.00284 & 0.00188 \\
    3 & 8 & 0.1 & 0.02417 & 0.02428 & 0.02041 \\
    3 & 16 & 0.001 & 5.74476E-05 & 5.53494E-05 & 0.00016 \\
    3 & 16 & 0.01 & 0.00031 & 0.00015 & 0.00095 \\
    3 & 16 & 0.1 & 0.02035 & 0.02036 & 0.01269 \\
    3 & 32 & 0.001 & \textbf{6.49630E-06} & \textbf{2.14812E-05} & \textbf{3.33377E-05} \\
    3 & 32 & 0.01 & \textbf{6.91262E-06} & 3.07683E-05 & \textbf{2.54072E-05} \\
    3 & 32 & 0.1 & 0.02028 & 0.01674 & 0.01653 \\
}

Al igual que en las anteriores secciones, se presentan las comparativas de los errores 
cuadráticos medios promedio para un número diferente de predicciones (1, 2 y 3 predicciones) del modelo 
estudiado y un parámetro diferente en cada ocasión.

En el caso del número de capas, se puede observar (Figura \ref{fig:Capas_MLP.png}) que a diferencia de
los otros modelos, la influencia del número de capas en la precisión es aparentemente menor.
\imagen{Capas_MLP.png}{Comparativa de número de capas para MLP}{0.9}

En cuanto al número de neuronas por capa, puede apreciarse que con 32 unidades se obtienen los modelos
con mayor precisión por término medio, siendo la configuración de 2 predicciones por entrada la que
generalmente arrojará resultados con menor error.
\imagen{Unidades_MLP.png}{Comparativa de número de unidades para MLP}{0.9}

Como puede observarse en la Figura \ref{fig:Lr_MLP.png}, el ratio de aprendizaje tiene una aparente
mayor influencia sobre los resultados, siendo la configuración de 0.001 la que menores errores por
término medio proporcionan.
\imagen{Lr_MLP.png}{Comparativa de ratio de aprendizaje para MLP}{0.9}

\section{Comparativa entre modelos}
\imagen{Capas_Comp.png}{Comparativa de número de capas}{0.9}
\imagen{Unidades_Comp.png}{Comparativa de número de unidades}{0.9}
\imagen{Lr_Comp.png}{Comparativa de ratio de aprendizaje}{0.9}