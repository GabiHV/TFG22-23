\capitulo{2}{Objetivos del proyecto}

El proyecto aborda diferentes objetivos. Podemos hacer una distinción entre 
objetivos generales, objetivos técnicos y objetivos personales.

\section{Objetivos generales}
\begin{itemize}
    \item Realizar un análisis de datos para ser capaces de predecir la temperatura
        y humedad del suelo en un viñedo.
    \item Facilitar la comprensión de los datos recogidos mediante representaciones
        gráficas.
    \item Buscar las correlaciones entre las diferentes medidas obtenidas y las 
        diferentes variables ambientales y/o de características del suelo.
    \item Encontrar modelos predictivos que permitan explicar el comportamiento
        de ciertas variables en función de las otras y anticipar la evolución futura
        a corto plazo.
\end{itemize}

\section{Objetivos técnicos}
\begin{itemize}    
    \item Realizar visualizaciones de los datos recogidos por los sensores IoT 
        (\textit{Internet of Things}) desplegados en el viñedo.
    \item Realizar un pre-procesamiento de datos mediante librerías de manipulación
        de datos de Python como Pandas.
    \item Modelar una Red Neuronal Artificial con Keras que permita predecir la humedad
        y temperatura del suelo gracias a los datos proporcionados.
    \item Emplear Git como sistema de control de versiones distribuido mediante
        la plataforma GitHub.
    \item Emplear ZenHub para la gestión de proyectos mediante las metodologías 
        ágiles.
    \item Aplicar en la medida de lo posible las metodologías ágiles mediante la 
        técnica Scrum aprendida durante el desarrollo del grado (ciertos aspectos
        como las "\textit{daily}" no pueden aplicarse debido al carácter del TFG).
\end{itemize}

\section{Objetivos personales}
\begin{itemize}
    \item Emplear los conocimientos y técnicas adquiridas durante el desarrollo de los diferentes
        cursos del Grado en Ingeniería Informática.
    \item Profundizar en el uso de la inteligencia artificial para la resolución de 
        problemas cotidianos que tendrá un reflejo en un sistema real (como es el
        campo del cultivo de la vid).
    \item Profundizar en la utilización de un lenguaje de programación tan versátil
        como es Python para el análisis de datos, empleando diferentes librerías
        nativas y de terceros, así como crear nuevos módulos.
    \item Profundizar en la librería de Tensorflow que se encuentra tan extendida
        en los diferentes ámbitos de la vida cotidiana (mundo laboral e investigador). 
\end{itemize}