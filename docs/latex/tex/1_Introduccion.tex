\capitulo{1}{Introducción}
El cuidado de las tierras de cultivo vitícolas ha supuesto a menudo a lo largo 
de la historia diferentes retos desde el comienzo de su domesticación entorno al 
3500-3100 a.C., continuando en la época del Antiguo Egipto, lugar en el que se cree se comenzó a popularizar la ingesta
del producto resultante de la labranza y cosecha de los cultivos: el vino \cite{book:piqueras2014}.

En la península ibérica el interés por este proceso y su resultante comenzó en el 
periodo fenicio cerca del 1100 a.C., expandiéndose por todo el territorio peninsular
gracias al comercio, lo que dio como resultado que los procesos enológicos fueran ya bien
conocidos en los siglos IV-III a.C. \cite{book:piqueras2014, MAPA:Historia}.

Entre los principales retos a los que se han enfrentado los viticultores en la península ibérica se 
encuentran numerosas plagas de insectos, ácaros, nematodos, diferentes vertebrados, moluscos, bacterias, etc \cite{book:mundi2004cap1}.
Siendo la más devastadora la causada por la histórica plaga de filoxera (\textit{Phylloxera vastatrix}) a 
finales del siglo XIX y comienzos del XX, que obligó a una reestructuración prácticamente completa de la viticultura española \cite{book:mundi2004cap3}.

Sin embargo, el principal factor influyente en el cultivo de la vid es el clima, siendo este el causante de sus alteraciones, fisiopatías y
la presencia de casi todas las plagas, de manera que estas se encuentran relacionadas con diferentes 
factores climáticos como la temperatura, las precipitaciones, la humedad y la humectación. 
De esta forma, por ejemplo, las altas lluvias o humedades y/o bajas temperaturas pueden producir corrimiento del racimo (escasez anormal de bayas en los racimos) o 
las altas temperaturas pueden generar desequilibrios hídricos que afecten a los injertos y a aquellas cepas infectadas por hongos de la madera \cite{book:mundi2004cap1}.

\clearpage

Por otro lado, los datos climatológicos históricos indican que la temperatura global ha aumentado 1,1ºC desde el 
inicio de la Revolución Industrial hasta nuestros días, siendo este desequilibrio causado en los últimos años
por la actividad humana, produciéndose a velocidad tal que los ecosistemas no tienen tiempo de adaptarse \cite{book:zuniga2021}.
Las precipitaciones, además, no son uniformes, sino que aumentan en zonas donde ya son muy abundantes y disminuyen
en las regiones más secas \cite{book:zuniga2021}.

Ante esta situación se plantea la cuestión de cuáles de estos efectos podríamos ser capaces de predecir
y en qué proporción para, si bien no subsanar completamente, si ser capaces de paliar en cierta medida
conociendo con anterioridad ciertos datos climatológicos que ayudarán a las diferentes tomas de decisiones de 
los viticultores.

De esta manera, siendo conocedores a corto o medio plazo la humedad y la temperatura superficial del suelo,
entre otras variables, podríamos adaptar los sistemas de riego para adecuarlos a las necesidades de los
cultivos y por otro lado, gestionar de forma adecuada los recursos hídricos nacionales (en este caso
de la zona hidrográfica del Duero) al mismo tiempo que protegemos a las vides de plagas y trastornos fisiológicos 
debidos a diferentes factores como la humectación o la temperatura del suelo.

En este proyecto se propone el análisis de los datos de temperatura y humedad del suelo procedentes de 
sensores IoT (\textit{Internet of Things}) desplegados en un viñedo situado en la zona sur de la provincia de Burgos, para, de esta forma,
obtener modelos predictivos que nos permitan conocer el comportamiento de ciertas variables en función de otras
y anticipar su evolución futura a corto plazo.

\section{Estructura de la memoria}
La memoria del proyecto cuenta con la siguiente estructura:
\begin{itemize}
    \item Introducción: descripción del problema que el proyecto pretende resolver junto con la estructuración 
        de la memoria y los materiales adjuntos.
    \item Objetivos del proyecto: descripción de los objetivos a cumplir con el desarrollo del proyecto.
    \item Conceptos teóricos: explicación de los conceptos teóricos necesarios para la comprensión del proyecto, 
        el problema a abordar y la solución propuesta.
    \item Técnicas y herramientas: listado de técnicas y herramientas empleadas durante el desarrollo del proyecto.
    \item Aspectos relevantes del desarrollo: aspectos a destacar durante la realización del proyecto.
    \item Trabajos relacionados: trabajos relacionados con el problema que aborda el proyecto.
    \item Conclusiones y líneas de trabajo futuras: conclusiones obtenidas de la realización del proyecto y posibilidades
        de ampliarlo o de introducir mejoras.
\end{itemize}

Además de la memoria, se incluyen una serie de anexos:
\begin{itemize}
    \item Plan de proyecto software: planificación temporal y estudio de viabilidad del proyecto.
    \item Comprensión del negocio: especificación de los objetivos y requisitos que el proyecto pretende abordar.
    \item Comprensión de los datos: estudio del conjunto de datos inicial y su adecuación al problema.
    \item Preparación de los datos: proceso llevado a cabo para seleccionar los datos empleados en el modelado.
    \item Modelado: creación de los diferentes modelos.
    \item Evaluación: evaluación de los modelos obtenidos.
    \item Manual del programador: aspectos relevantes del código fuente del proyecto.
    \item Manual de usuario: guía de usuario para el manejo de la aplicación asociada al proyecto
\end{itemize}

\section{Materiales adjuntos}
El proyecto incluye el siguiente contenido:
\begin{itemize}
    \item Python Notebook con el pre-procesamiento de datos y las diferentes explicaciones.
    \item Python Notebook con la integración de los diferentes conjuntos de datos en un único fichero.
    \item Python Notebook con el modelado de la Red Neuronal Artificial empleada para la regresión
        de humedad y temperatura y las explicaciones correspondientes.
    \item Conjunto de datos de los sensores desplegados en el viñedo.
    \item Conjunto de datos del pluviómetro desplegado en el viñedo.
    \item Scripts de Python para mostrar las gráficas de los datos (de sensores y pluviómetro) sin procesar.
    \item Scripts de Python para mostrar las gráficas de los datos (de sensores y pluviómetro) procesados
    \item Script de Linux para la instalación del entorno virtual.
    \item Script de Windows PowerShell para la instalación del entorno virtual.
    \item Script de Windows CMD para la instalación del entorno virtual.
\end{itemize}
Los recursos mencionados se encuentran disponibles en GitHub \cite{gabriel:2023}.
