\capitulo{1}{Introducción}
El cuidado de las tierras de cultivo vitícolas ha supuesto a menudo a lo largo 
de la historia diferentes retos desde el comienzo de su domesticación entorno al 
3500-3100 a.C., continuando en la época del Antiguo Egipto (donde se han hallado 
cantidad de pinturas en las que se describen con grandes detalles el proceso
del cuidado de la vid), lugar en el que se cree se comenzó a popularizar la ingesta
del producto resultante de la labranza y cosecha de los cultivos: el vino \cite{alma991002270696805771}.

En la península iberica el interés por este proceso y su resultante comenzó en el 
periodo fenicio cerca del 1100 a.C., expandiéndose por todo el territorio peninsular
gracias al comercio, lo que dió como resultado que los procesos enológicos fueran ya bien
conocidos en los siglos IV-III a.C. \cite{alma991002270696805771, MAPA_Historia}.


\section{Estructura de la memoria}
La memoria del proyecto cuenta con la siguiente estructura:
\begin{itemize}
    \item Introducción:
    \item Objetivos del proyecto:
    \item Conceptos teóricos:
    \item Técnicas y herramientas:
    \item Aspectos relevantes del desarrollo:
    \item Trabajos relacionados:
    \item Conclusiones y líneas de trabajo futuras:
\end{itemize}

\section{Materiales adjuntos}


