\capitulo{1}{Introducción}
El cuidado de las tierras de cultivo vitícolas ha supuesto a menudo a lo largo 
de la historia diferentes retos desde el comienzo de su domesticación entorno al 
3500-3100 a.C., continuando en la época del Antiguo Egipto (donde se han hallado 
cantidad de pinturas en las que se describen con grandes detalles el proceso
del cuidado de la vid), lugar en el que se cree se comenzó a popularizar la ingesta
del producto resultante de la labranza y cosecha de los cultivos: el vino \cite{book:piqueras2014}.

En la península ibérica el interés por este proceso y su resultante comenzó en el 
periodo fenicio cerca del 1100 a.C., expandiéndose por todo el territorio peninsular
gracias al comercio, lo que dió como resultado que los procesos enológicos fueran ya bien
conocidos en los siglos IV-III a.C. \cite{book:piqueras2014, MAPA:Historia}.

Entre los principales retos a los que se han enfrentado los viticultores en la península ibérica se 
encuentran numerosas plagas de insectos, ácaros, nematodos, diferentes vertebrados, moluscos, bacterias, etc \cite{book:mundi2004}.
Siendo la más devastadora la causada por la histórica palga de filoxera (\textit{Phylloxera vastatrix}) a 
finales del siglo XIX y comienzos del XX, que obligó a una reestructuración prácticamente completa de la viticultura española \cite{book:mundi2004}.

Sin embargo, el principal factor influyente en el cultivo de la vid es el clima, siendo este el causante de sus alteraciones, fisiopatías y
la presencia de casi todas las plagas, de manera que estas se encuentran relacionadas con diferentes 
factores climáticos como la temperatura, las precipitaciones, la humedad y la humectación. 
De esta forma, por ejemplo, las altas lluvias o humedades y/o bajas temperaturas pueden producir corrimiento del racimo (escasez anormal de bayas en los racimos) o 
las altas temperaturas pueden generar desequilibrios hídricos que afecten a los injertos y a aquellas cepas infectadas por hongos de la madera \cite{book:mundi2004}.

\section{Estructura de la memoria}
La memoria del proyecto cuenta con la siguiente estructura:
\begin{itemize}
    \item Introducción: descripción del problema que el proyecto pretende resolver junto con la estructuración 
        de la memoria y los materiales adjuntos.
    \item Objetivos del proyecto: descripción de los objetivos a cumplir con el desarrollo del proyecto.
    \item Conceptos teóricos: explicación de los conceptos teóricos necesarios para la comprensión del proyecto, 
        el problema a abordar y la solución propuesta.
    \item Técnicas y herramientas: listado de técnicas y herramientas empleadas durante el desarrollo del proyecto.
    \item Aspectos relevantes del desarrollo: aspectos a destacar durante la realización del proyecto.
    \item Trabajos relacionados: trabajos relacionados con el problema que aborda el proyecto.
    \item Conclusiones y líneas de trabajo futuras: conclusiones obtenidas de la realización del proyecto y posibilidades
        de ampliarlo o de introducir mejoras.
\end{itemize}

Además de la memoria, se incluyen una serie de anexos:
\begin{itemize}
    \item Plan de proyecto software: planificación temporal y estudio de viabilidad del proyecto.
    \item Especificación de los requisitos del software: descripción de la fase de análisis (objetivos y requisitos funcionales y no funcionales).
    \item Especificación del diseño: descripción de la fase del diseño.
    \item Manual del programador: aspectos relevantes del código fuente del proyecto.
    \item Manual de usuario: guía de usuario para el manejo de la aplicación asociada al proyecto
\end{itemize}

% \section{Materiales adjuntos}
% El proyecto incluye 


