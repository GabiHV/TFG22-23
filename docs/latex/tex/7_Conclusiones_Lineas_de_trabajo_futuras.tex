\capitulo{7}{Conclusiones y Líneas de trabajo futuras}
En esta sección se presentan las conclusiones derivadas del desarrollo del trabajo, así como las
líneas de trabajo futuras con las que se podría dar continuidad al proyecto.

\section{Conclusiones}
En lo referente a las aspectos generales del desarrollo del proyecto, se pueden extraer las siguientes
conclusiones:
\begin{itemize}
    \item Los objetivos del proyecto se han llevado a término de forma correcta, cumpliendo,
        por otro lado, los requisitos impuestos durante el desarrollo de este. La predicción 
        de los valores futuros de la serie temporal es funcional y arroja errores que pueden
        calificarse como aceptables en ciertos modelos y con los datos proporcionados.
    \item Las entradas y salidas de los modelos se han parametrizado, de modo que puedan variarse
        el número de ejemplares utilizados para las predicciones o la propia cantidad de valores
        futuros. Por otro lado, también es posible modificar otros parámetros como el ratio de 
        aprendizaje, las épocas o el tamaño del lote de entrenamiento.
    \item Además de parametrizar aspectos del diseño de los modelos, se permite variar la cantidad
        de datos con la que se entrenan, validan y comprueban, de forma que el proceso llevado a cabo
        durante el desarrollo pueda emplearse en trabajos futuros o en posibles cambios de diseño
        en el \textit{KDD}.
    \item Durante el proyecto se han empleado diferentes librerías para diversos propósitos, como
        Pandas en el caso de la manipulación de grandes conjuntos de datos o Keras para realizar
        los distintos modelados, así como módulos nativos del lenguaje de programación empleado. 
    \item A pesar del incidente surgido con el sistema de gestión del proyecto ZenHub, se ha 
        profundizado y continuado en la medida de lo posible con la aplicación
        de las metodologías ágiles en el desarrollo de proyectos, utilizando como proceso de 
        desarrollo los incrementos iterativos (\textit{sprint}) cada dos semanas.
    \item La investigación llevada a cabo permite mejorar los procesos y los conocimientos adquiridos
        en la búsqueda de bibliografía y la lectura de artículos científicos de los campos relacionados
        con el desarrollo.
\end{itemize}

Respecto a los modelos obtenidos, se puede concluir que para los datos proporcionados y los hiperparámetros
empleados, las redes neuronales recurrentes tienen un desempeño mayor que los modelos neuronales
convencionales para predicciones de series de tiempo, destacando \textit{GRU}
entre los tres del ensayo y siendo la variante de 2 capas, 32 celdas de memoria y un \textit{learning rate}
de 0.01 la que mejor precisión proporciona, obteniéndose, de igual forma, mejores resultados en 
promedio en arquitecturas \textit{LSTM}.

La Tabla \ref{tabla:Resumen de ejecuciones} muestra los 4 mejores resultados por modelo (que se 
corresponden en todas las situaciones a arquitecturas con una única predicción) junto con el 
promedio total de los resultados de cada uno.
\tablaSmallSinColores{Resumen de ejecuciones}{l l l l}{Resumen de ejecuciones}{& \textbf{GRU} & \textbf{LSTM} & \textbf{MLP}\\}{
  & \textbf{2.69205E-07} & 4.56945E-07 & 5.56436E-06\\
  & 4.66034E-07 & 6.54021E-07 & 5.57840E-06 \\
  & 5.66822E-07 & 6.62421E-07 & 6.49630E-06 \\
  & 5.96471E-07 & 9.51891E-07 & 6.91262E-06 \\
  \hline
  Promedio Total & 2.57000E-03 & \textbf{5.64000E-04} & 0.00736 \\ 
}

\section{Líneas de trabajo futuras}
El desarrollo del proyecto podría continuarse con diferentes líneas de trabajo, que por su naturaleza
no caben en el planteamiento inicial, ya sea porque no se han previsto o porque el tiempo
de desarrollo es más elevado del disponible.

A continución se presentan un conjunto de posibles líneas futuras:
\begin{itemize}
    \item Emplear una combinación de los modelos propuestos en el proyecto con redes convolucionales
        para introducir una componente espacial en los datos y, de esta forma, ser capaces de predecir valores
        en una situación concreta del viñedo.
    \item Emplear los datos de la variación de las precipitaciones para realizar un estudio de la precisión
        de los modelos con este nuevo atributo. De esta forma, se podría mejorar el rendimiento obtenido
        en variables como las humedades. 
    \item Para conocer el desempeño de los resultados se puede desarrollar una comparativa de las precisiones
        con datos extraidos de otros viñedos. De modo que se observaría si los modelos son extrapolables
        a otras situaciones o si en cambio se tratan de resultados específicos.
    \item Realizar una integración completa en un producto software de cualquier índole relacionado con
        la empresa de la que se han extraido los datos, véase, por ejemplo, la automatización de los sistemas de riego
        instalados. En este ejemplo se desarrollaría un sistema de toma de decisiones basado en los resultados obtenidos
        con los modelos implementados.
    \item En lo referente al propio resultado del proyecto, para que los usuarios con menos conocimientos puedan manipular los
        hiperparamámetros de los modelos, seleccionar los datos deseados y realizar visualizaciones más guiadas de los resultados, 
        podría implementarse una interfaz que ocultara la complejidad del código fuente final.
\end{itemize}