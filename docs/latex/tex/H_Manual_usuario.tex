\apendice{Documentación de usuario}

\section{Introducción}
En esta sección se presentará la manera de cargar los modelos resultantes en un fichero Python para poder ser desplegados en un producto software, 
así como la forma que deberá tener el tensor de entrada a la red y los datos de salida.

\section{Requisitos de usuarios}
En cuanto a los requisitos del usuario, será necesario tener instalado \textbf{Python 3.9.13} \cite{misc:python2023}, junto con la versión 2.11.0 de \textbf{TensorFlow} \cite{misc:tensorflow2023},
de manera que se puedan cargar los modelos almacenados empleando la función de la \textit{API} \textbf{Keras} para tal propósito.

\section{Instalación}
En cuanto a la instalación de la versión concreta del intérprete de Python puede realizarse en \cite{misc:python2023}, mientras que para instalar la dependencia de la \textit{API},
se puede introducir el comando:
\begin{lstlisting}[language=Bash]
    pip3 install tensorflow=2.11.0
\end{lstlisting}

Para cargar un modelo con la librería mencionada, se debe emplear \cite{misc:tensorflow_save2023}:
\begin{lstlisting}[language=Python]
    from tensorflow import keras
    keras.models.load_model('<path_del_modelo>')
\end{lstlisting}

\section{Manual del usuario}
Para realizar predicciones con el modelo pertinente, las entradas deben tener una estructura concreta que dependerá de cómo se haya entrenado a cada una de las diferentes redes neuronales.
Es decir, el tamaño del número de muestras de entrada dependerá de la cantidad de ``\textit{backtracking}'' que se haya establecido en el entrenamiento.

En este caso se debe introducir un tensor bidimensional de 6 muestras (se han obtenido modelos que aceptan 6 horas) con 7 variables de entrada cada una que se corresponde a:
\begin{itemize}
    \item \textbf{t\_ext:} temperatura exterior media en una hora (-50, 50).
    \item \textbf{h\_ext:} humedad exterior media en una hora (0, 100).
    \item \textbf{t\_C\_cal:} temperatura media de la sonda de temperatura más superficial en una hora (-50, 50).
    \item \textbf{h\_C\_cal:} humedad media de la sonda de humedad más superficial en una hora (0, 100).
    \item \textbf{t\_L\_cal:} temperatura media de la sonda de temperatura interna en una hora (-50, 50).
    \item \textbf{h\_L\_cal:} humedad media de la sonda de humedad interna en una hora (0, 100).
    \item \textbf{sensor:} sensor al que se corresponde los datos (0, 7).
\end{itemize}

Por otro lado, las variables deberán estar normalizadas en el rango 0-1, con los máximos y mínimos especificados anteriormente.

En el caso del modelo \textit{MLP}, en lugar de un tensor tridimensional, se deberá modificar para que se corresponda con un vector del número de variables por el de muestras.

La salida será igualmente un tensor bidimensional del número de predicciones establecidas en el entrenamiento del modelo con 6 variables cada una que se corresponden a las mencionadas
anteriormente a excepción del número de sensor. 

De forma inversa, el modelo proporcionará datos normalizados, por lo que para obtener cada atributo en un rango correcto deberá de denormalizarse.
