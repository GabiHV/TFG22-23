\apendice{Comprensión de los datos}

\section{Introducción}
En la etapa de comprensión de los datos se creará el conjunto inicial y se comprobará
si este es adecuado, para, en caso contrario, poder seguir recopilando.

Los plazos del proyecto impiden realizar esta fase en toda su extensión, puesto que
a pesar de ser el proceso de análisis de datos un proceso iterativo, se encuentra 
sujeto a unos plazos inamovibles.

\section{Descripción de los datos}
Se proporcionan un conjunto de datos compuesto por varios sensores (en concreto 8 
sensores \textit{IoT}) y un pluviómetro desplegados en un viñedo.

Los datos de los sensores proporcionados cuentan con los siguientes atributos:

\tablaSmallSinColores{Atributos de los datos: sensores}{m{5em} m{15em} m{10em}}{Atributos sensores}{\textbf{Atributo} & \textbf{Descripción}  & \textbf{Medida} \\}
{
ts          & \textit{Timestamp} de la muestra                    & ms \\
fecha       & Fecha de la muestra                                 & Fecha en formato yy/mm/dd, hh:mm:ss \\
batería     & Nivel de batería del sensor                         & V \\
t\_ext      & Temperatura ambiental                               & ºC \\
h\_ext      & Humedad ambiental                                   & \% \\
t\_C\_cal   & Temperatura superficial a 20cm.                     & ºC \\
h\_C\_cal   & Humedad superficial a 20cm.                         & \% \\
t\_L\_cal   & Temperatura a profundidad mayor a t\_C\_cal         & ºC \\
h\_L\_cal   & Humedad superficial a profundiad mayor a h\_C\_cal  & \% \\
h\_C        & Humedad superficial sin calibrar                    & Valor relacionado con capacitancias \\
h\_L        & Humedad profunda sin calibrar                       & Valor relacionado con capacitancias \\
}

Los datos del pluviómetro cuentan con los siguientes atributos:
\tablaSmallSinColores{Atributos de los datos: pluviómetro}{m{7em} m{13em} m{10em}}{Atributos pluviómetro}{\textbf{Atributo} & \textbf{Descripción}  & \textbf{Medida} \\}
{
ts              & \textit{Timestamp} de la muestra                      & ms \\
fecha           & Fecha de la muestra                                   & Fecha en formato yy/mm/dd, hh:mm:ss \\
batería         & Nivel de batería del sensor                           & V \\
pluv            & Contador del incremento de activación del balancín    & N/A \\
pluv\_delta     & Incremento de activación del balancín                 & N/A \\
pluv\_deltaMM   & Precipitaciones registradas desde la última lectura   & litros/\(m^2\) \\
}

\subsection{Visualización de los datos}
Para dar una idea del estado de los datos se proporcionan unas gráficas de dispersión
con las lecturas realizadas en cada uno de los sensores y el pluviómetro.
En estas podrán observarse los valores \textbf{sin procesar} de cada atributo 
en función del tiempo.

En el caso de los sensores, los atributos \textit{h\_C} y \textit{h\_L} se obviarán debido
a que se tratan de datos redundantes con las respectivas humedades calibradas. 
Además, los datos de batería y fecha tampoco se emplearán en la creación de las gráficas.

% Sensor 1
\imagen{raw_data/sensor1.png}{Datos no procesados: sensor 1}{1}

% Sensor 2
\imagen{raw_data/sensor2.png}{Datos no procesados: sensor 2}{1}

% Sensor 3
\imagen{raw_data/sensor3.png}{Datos no procesados: sensor 3}{1}

% Sensor 4
\imagen{raw_data/sensor4.png}{Datos no procesados: sensor 4}{1}

% Sensor 5
\imagen{raw_data/sensor5.png}{Datos no procesados: sensor 5}{1}

% Sensor 6
\imagen{raw_data/sensor6.png}{Datos no procesados: sensor 6}{1}

% Sensor 7
\imagen{raw_data/sensor7.png}{Datos no procesados: sensor 7}{1}

% Sensor 8
\imagen{raw_data/sensor8.png}{Datos no procesados: sensor 8}{1}

% Pluviómetro
\imagen{raw_data/pluviometro.png}{Datos no procesados: pluviómetro}{1}