\apendice{Comprensión de los datos}

\section{Introducción}
En la etapa de comprensión de los datos se creará el conjunto inicial y se comprobará
si este es adecuado, para, en caso contrario, poder continuar recopilando.

Los plazos del proyecto impiden realizar esta fase en toda su extensión, puesto que
a pesar de ser el proceso de análisis de datos un proceso iterativo, se encuentra 
sujeto a unos plazos inamovibles.

\section{Descripción de los datos}
Se proporcionan un conjunto de datos compuesto por varios sensores (en concreto 8 
sensores \textit{IoT}) y un pluviómetro desplegados en un viñedo.

Los datos de los sensores proporcionados cuentan con los siguientes atributos:

\tablaSmallSinColores{Atributos de los datos: sensores}{m{5em} m{15em} m{10em}}{Atributos sensores}{\textbf{Atributo} & \textbf{Descripción}  & \textbf{Medida} \\}
{
ts          & \textit{Timestamp} de la muestra                    & ms \\
fecha       & Fecha de la muestra                                 & Fecha en formato yy/mm/dd, hh:mm:ss \\
batería     & Nivel de batería del sensor                         & V \\
t\_ext      & Temperatura ambiental                               & ºC \\
h\_ext      & Humedad ambiental                                   & \% \\
t\_C\_cal   & Temperatura superficial a 20cm.                     & ºC \\
h\_C\_cal   & Humedad superficial a 20cm.                         & \% \\
t\_L\_cal   & Temperatura a profundidad mayor a t\_C\_cal         & ºC \\
h\_L\_cal   & Humedad superficial a profundiad mayor a h\_C\_cal  & \% \\
h\_C        & Humedad superficial sin calibrar                    & Valor relacionado con capacitancias \\
h\_L        & Humedad profunda sin calibrar                       & Valor relacionado con capacitancias \\
}

Los datos del pluviómetro cuentan con los siguientes atributos:
\tablaSmallSinColores{Atributos de los datos: pluviómetro}{m{7em} m{13em} m{10em}}{Atributos pluviómetro}{\textbf{Atributo} & \textbf{Descripción}  & \textbf{Medida} \\}
{
ts              & \textit{Timestamp} de la muestra                      & ms \\
fecha           & Fecha de la muestra                                   & Fecha en formato yy/mm/dd, hh:mm:ss \\
batería         & Nivel de batería del sensor                           & V \\
pluv            & Contador del incremento de activación del balancín    & N/A \\
pluv\_delta     & Incremento de activación del balancín                 & N/A \\
pluv\_deltaMM   & Precipitaciones registradas desde la última lectura   & litros/\(m^2\) \\
}

\section{Visualización de los datos}
Para dar una idea del estado de los datos se proporcionan unas gráficas de dispersión
con las lecturas realizadas en cada uno de los sensores y el pluviómetro.
En estas podrán observarse los valores \textbf{sin procesar} de cada atributo 
en función del tiempo.

En el caso de los sensores, los atributos \textit{h\_C} y \textit{h\_L} se obviarán debido
a que se tratan de datos redundantes con las respectivas humedades calibradas. 
Además, los datos de batería y fecha tampoco se emplearán en la creación de las gráficas,
puesto que no aportan información útil para los objetivos del proyecto más allá de posibles
errores de muestreo.

\subsection{Sensor 1}

% Sensor 1
\imagen{raw_data/sensor1.png}{Datos no procesados: sensor 1}{1}

En el primer sensor, como puede observarse en la Figura \ref{fig:raw_data/sensor1.png},
se encuentran varias situaciones que a \textit{priori} son sencillas de solucionar, 
y es que existe una cantidad importante de muestras para las que no contamos con datos,
además de existir valores claramente inverosímiles (como en el caso de la humedad
más profunda, o el aumento tan repentino de la humedad superficial).
A pesar de esto, el resto parecen estar dentro de la normalidad, por lo que
el trabajo en el sensor mencionado será uno de los menos tediosos.

Las subidas tan repentinas como las observadas en la imágen posteriormente se demostró
que eran debidas al agotamiento de la batería del respetivo sensor, que poco antes
de producirse arrojaban lecturas artificiales para luego realizar lecturas nulas.

\subsection{Sensor 2}

% Sensor 2
\imagen{raw_data/sensor2.png}{Datos no procesados: sensor 2}{1}

En el segundo caso, además de las lecturas para las que no contemos con valores,
el la Figura \ref{fig:raw_data/sensor2.png} se observa cómo llegado un instante
las sondas de temperatura y humedad superficiales comienzan a realizar lecturas
muy variadas a lo largo del tiempo, para posteriormente dejar de producir datos.

Además, puede observarse la presencia de cierto ruido durante los muestreos previos
al suceso mencionado en las sondas de humedad, lo que posiblemente sea debido a 
las capacitancias de estas.

% Una investigación posterior confirmó que el sensor había sido desplazado de su 
% posición original debido probablemente a las labores realizadas en los viñedos.

\newpage
\subsection{Sensor 3}

% Sensor 3
\imagen{raw_data/sensor3.png}{Datos no procesados: sensor 3}{1}

En el este sensor, como puede observarse en la Figura \ref{fig:raw_data/sensor3.png},
además de los muestras para los que no tenemos valores (que será la tónica habitual
en el conjunto de datos), existen unas subidas remarcables de humedades, con unas
bajadas que pueden considerarse aceleradas o repentinas.

La mayor parte de estas son genuinas y debidas a las lluvias contrastadas esos días, sin 
embargo, la última de las subidas se observa contextualmente errónea, provocando 
un cambio brusco en ambas humedades del suelo.

\newpage
\subsection{Sensor 4}

% Sensor 4
\imagen{raw_data/sensor4.png}{Datos no procesados: sensor 4}{1}

En este caso, pueden observarse en la Figura \ref{fig:raw_data/sensor4.png}, diferentes
situaciones. La primera de ellas, como en el resto de sensores son los valores faltantes; la
segunda es el ruido existente en ciertos periodos de tiempo.

Tras este ruido instantáneo en los datos, puede observarse cómo la variabilidad de las muestras
aumenta de forma homogénea en todas y cada una de las sondas, a salvedad de las que recogen
datos ambientales.

Las posteriores investigaciones indicaron que de nuevo el sensor había sido desplazado
de su situación inicial.

\newpage
\subsection{Sensor 5}

% Sensor 5
\imagen{raw_data/sensor5.png}{Datos no procesados: sensor 5}{1}

En la Figura \ref{fig:raw_data/sensor5.png} puede observarse diferentes situaciones.
Una de ellas es la repetida falta de valores para algunos instántes de tiempo; la 
segunda es las aparentes subidas repentinas de humedades.

Estas siguen el mismo patrón en todas las situaciones, aumento que puede clasificarse
como repentino, y un descenso moderado a lo largo del tiempo.
De esta forma, todas se contrastaron como genuinas observando ciertas situaciones como
el tiempo atmosférico en los instántes de subida.

Este sensor se trata de uno con los datos más limpios de forma predefinida de todo
el conjunto de datos, sino el que en mejor estado se encuentra.

\newpage
\subsection{Sensor 6}

% Sensor 6
\imagen{raw_data/sensor6.png}{Datos no procesados: sensor 6}{1}

En el conjunto de datos que se puede observar en la Figura \ref{fig:raw_data/sensor6.png},
al igual que en los anteriores sensores, pueden observarse valores faltantes, además de 
otras situaciones.

Una de ella es la presencia generalizada de ruido en el conjunto de ejemplos en las lecturas
de las sondas que miden variables del suelo.

Por otro lado, en un instante de tiempo las humedades aumentan de forma repentina, posiblemente
por la lluvia, sin embargo en la humedad del suelo a mayor profundidad, hay una variabilidad
que se consideró como irrecuperable por su estado.

\newpage
\subsection{Sensor 7}

% Sensor 7
\imagen{raw_data/sensor7.png}{Datos no procesados: sensor 7}{1}

En este caso pueden observarse en la Figura \ref{fig:raw_data/sensor7.png} diferentes
situaciones, además de la falta ya mencionada en el resto de sensores de valores
en algunas muestras.

En primer lugar, en las sondas de humedad no ambientales se comienza a producir un aumento
que podría considerarse como exponencial en el tiempo debido al agotamiento de la batería.

En segundo lugar, como en el sensor 4, se produce un aumento significativo de la variabilidad
de la temperatura del suelo, produciéndose, además, un aumento de la sonda de
humedad más profunda.

De igual forma que en el sensor mencionado, las labores en los cultivos habían producido un
cambio en el emplazamiento original del sensor, lo que producía estas situaciones mencionadas.
Con respecto a la falta de los datos de humedad y temperatura más superficiales se 
comprobará que fue debida a la rotura de las respectivas sondas.

\newpage
\subsection{Sensor 8}

% Sensor 8
\imagen{raw_data/sensor8.png}{Datos no procesados: sensor 8}{1}

Como puede observarse en la figura \ref{fig:raw_data/sensor8.png}, en este caso existen 
múltiples problemas en el conjunto de datos.

En primer lugar, los ya mencionados valores faltantes en diferentes instantes de tiempo.
En segundo lugar, los aumentos de temperatura debidos al agotamiento de la batería y en tercer
lugar la alta variabilidad en las temperaturas, que por otro lado, se asemejan en gran medida,
lo que indica que de nuevo el sensor había sido desplazado de su emplazamiento original.

Este será uno de los sensores con el conjunto de datos con más problemas de todos los 
expuestos anteriormente.

\newpage
\subsection{Pluviómetro}

% Pluviómetro
\imagen{raw_data/pluviometro.png}{Datos no procesados: pluviómetro}{1}

En el caso del pluviómetro, los datos estarán sujetos a muestras erróneas debido a la forma
de fijación del mismo, puesto que con las ráfagas de viento se producían oscilaciones
en el poste al que se encontraba anclado, de forma que se realizaban lecturas incorrectas
de las precipitaciones.
