\apendice{Preparación de los datos}

\section{Introducción}
En la fase de preparación de los datos se realiza la limpieza de los mismos para 
que posteriormente puedan ser empleados por los modelos y obtener resultados adecuados
para ser utilizados en un posterior despliegue.

Esta fase está compuesta por la detección y el tratamiento de varias casuísticas en 
los datos, como los valores faltantes, que se tratan de valores de los que no tenemos
registros por diferentes motivos y los valores extremos, que son aquellos
estadísticamente lejanos al resto, lo que no quiere decir que sean erróneos.

Por otro lado, también se pueden encontrar valores erróneos, que pueden ser 
estadísticamente correctos, pero son datos que no tienen sentido contextual, en este
caso concreto se tratará de subidas o bajadas inverosímiles de temperaturas 
y/o humedades del suelo, que se encuentran dentro de los valores naturales,
pero que no se pueden producir a la velocidad a la que tienen lugar.
En este caso, se empleará una exploración visual para realizar su tratamiento. 

Cabe destacar que a todos los conjuntos de datos se les ha aplicado un filtro de 
tendencias (en este caso el filtro de Hodrick-Presscot \cite{misc:wikipediaHodrick2023}), 
puesto que no existe un interés específico en los datos concretos, sino en las posibles 
tendencias de las diferentes variables, que nos permitan obtener una predicción adecuada 
en cada circunstancia.
De esta manera, se ha establecido la tendencia diaria de los datos.

\section{Valores faltantes}
En cuanto a los valores faltantes, la decisión fue elimiar las muestras que tuvieran
indeterminados en alguna de sus columnas, puesto que la recuperación empleando técnicas
como la imputación de valores era prácticamente inviable y por otro lado, se contaban
con ejemplos suficientes como para poder suprimir parte de estos.

\section{Valores extremos}
En cuanto a la detección y el tratamiento de los valores faltantes, se empleó el 
rango intercuartílico para realizar la primera de las tareas, sustituyendo las variables
del atributo concreto de los ejemplos detectados con la mediana diaria.

\imagen{iqr.png}{Representación gráfica del método del rango intercuartílico}{0.75}

De esta manera se eliminaba el ruido presente en alguno de los sensores, a la par 
que no se producían cambios bruscos en el conjunto de datos gracias a emplear la 
mediana del grupo.

\section{Valores erróneos}
En lo referente a los valores erróneos se trata de una de las tareas más arduas 
del proyecto, puesto que parte de este se desarrolló de forma visual diréctamente
sobre el conjunto de datos de los diferentes sensores y el pluviómetro, siendo, 
de esta manera, en su mayoría un proceso manual.

Para tal propósito se estableció una columna adicional al conjunto de datos original
que indicaba la validez de la muestra concreta.

\section{Visualización de los datos}
En esta sección se explicará el tratamiento realizado en cada uno de los sensores, y,
de esta forma, poner en contexto cómo se han resuelto los problemas detectados
en la fase anterior.

\subsection{Sensor 1}
% Sensor 1
\imagen{processed_data/sensor1.png}{Datos procesados: sensor 1}{1}

Como puede observarse en la Figura \ref{fig:processed_data/sensor1.png}, 
se ha tratado el ruido presente en el conjunto de datos original, sustituyendo los 
respectivos valores por la mediana diaria.

Por otro lado, se ha determinado la supresión de las muestras que indicaban una 
variabilidad brusca de las humedades del suelo debidas a el agotamiento de la batería.
Esta decisión estuvo motivada por la pérdida moderada de datos, además de la 
falta de las muestras siguientes.

\newpage

\subsection{Sensor 2}
% Sensor 2
\imagen{processed_data/sensor2.png}{Datos procesados: sensor 2}{1}

En este caso, como puede observarse en la Figura \ref{fig:processed_data/sensor2.png},
se aplicó de nuevo la detección y tratamiento de ruido empleando la mediana como 
valor sustitutorio, además, ante la aparente imposibilidad de obtener una adecuada
recuperación de los datos con alta variabilidad, se optó por no utilizarlos.

\newpage

\subsection{Sensor 3}
% Sensor 3
\imagen{processed_data/sensor3.png}{Datos procesados: sensor 3}{1}

Como puede observarse en la Figura \ref{fig:processed_data/sensor3.png}, en este caso
concreto, el conjunto de datos se modifica, en cuanto a la invalidez de ejemplos se refiere,
visiblemente menos que los sensores anteriores y los siguientes.

En este caso solo fue necesario retirar del conjunto de datos una porción moderada de las
muestras totales que había resultado del agotamiento de la batería del sensor.

\newpage

\subsection{Sensor 4}
% Sensor 4
\imagen{processed_data/sensor4.png}{Datos procesados: sensor 4}{1}

En este sensor se optó, como refleja la Figura \ref{fig:processed_data/sensor4.png}, 
por la eliminación de los datos a partir del comienzo de las lecturas que ya
se conocían que eran incorrectas, lo que dejaba un conjunto de datos preparado
para ser empleado por los modelos. 

\newpage

\subsection{Sensor 5}
% Sensor 5
\imagen{processed_data/sensor5.png}{Datos procesados: sensor 5}{1}

En este caso, a diferencia del resto de sensores, no fue necesario una modificación
tan exhaustiva del conjunto de datos, puesto que los únicos tratamientos fueron la 
detección de \textit{outliers} o valores extremos y la eliminación de los ejemplos
con valores faltantes.

\newpage

\subsection{Sensor 6}
% Sensor 6
\imagen{processed_data/sensor6.png}{Datos procesados: sensor 6}{1}

En el caso del sensor 6, como puede apreciarse en la Figura 
\ref{fig:processed_data/sensor6.png}, se ha tratado y eliminado el ruido presente 
mediante la detección de valores extremos. Además, la anomalía de la humedad ha sido
eliminada al tratarse de una situación que no podía corregirse, por ejemplo, con los
métodos mencionados.

\newpage

\subsection{Sensor 7}
% Sensor 7
\imagen{processed_data/sensor7.png}{Datos procesados: sensor 7}{1}

En este caso existían dos problemas principales en el conjunto de datos original:
las sondas de temperatura y humedad comenzaban a realizar lecturas con variaciones
muy elevadas para su profundidad en cortos periodos de tiempo y a la par
existían atributos que no contaban con valores, además, al igual que en otros casos,
el agotamiento de la batería producía lecturas incorrectas en las humedades.

Ante estos problemas las soluciones tomadas fueron las de eliminar los ejemplos 
afectados. En el primer caso porque se trataban de periodos de tiempo moderados y 
en el segundo porque se había producido el desplazamiento del sensor de su posición 
original.

\newpage

\subsection{Sensor 8}
% Sensor 8
\imagen{processed_data/sensor8.png}{Datos procesados: sensor 8}{1}

Como puede observarse en la Figura \ref{fig:processed_data/sensor8.png}, en este caso
no ha sido posible recuperar los datos de gran parte del periodo muestreado.
Los aumentos de humedad mostrados en la figura indicada están contrastados como 
provocados por las altas precipitaciones.

Hablándo en términos de sesgos, es posible que al realizar la recuperación de estos
valores en los sensores mencionados, puedan haberse sesgado en cierta medida
los conjuntos de datos, sin embargo, los datos no dejarán de estar más cercanos a los
valores reales y, por tanto, contener un error más reducido que las muestras originales
con esta casuística.