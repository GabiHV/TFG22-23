\apendice{Preparación de los datos}

\section{Introducción}
En la fase de preparación de los datos se realiza la limpieza de los mismos para 
que posteriormente puedan ser empleados por los modelos y obtener resultados adecuados
para ser empleados en un posterior despliegue.

Esta fase está compuesta por la detección y el tratamiento de varias casuísticas en 
los datos, como los valores faltantes, que se tratan de valores de los que no tenemos
registros por diferentes motivos y los valores extremos, que son aquellos
estadísticamente lejanos al resto, lo que no quiere decir que sean erróneos.

Por otro lado, también se pueden encontrar valores erróneos, que pueden ser 
estadísticamente correctos, pero son datos que no tienen sentido contextual, en este
caso concreto se tratará de subidas o bajadas inverosímiles de temperaturas 
y/o humedades del suelo, que se encuentran dentro de los valores naturales,
pero que no se pueden producir a la velocidad a la que tienen lugar.
En este caso, se empleará una exploración visual para realizar su tratamiento. 

\section{Valores faltantes}
En cuanto a los valores faltantes, la decisión fue elimiar las muestras que tuvieran
indeterminados en alguna de sus columnas, puesto que la recuperación empleando técnicas
como la imputación de valores era prácticamente inviable y por otro lado, se contaban
con ejemplos suficientes como para poder suprimir parte de estos.

\section{Valores extremos}
En cuanto a la detección y el tratamiento de los valores faltantes, se empleó el 
rango intercuertílico para realizar la primera de las tareas, sustituyendo las variables
del atributo concreto de los ejemplos detectados con la mediana diaria.

%% Explicar gráficamente IQR1.75

De esta manera se eliminaba el ruido presente en alguno de los sensores, a la par 
que no se producían cambios bruscos en el conjunto de datos gracias a emplear la 
mediana del grupo.

\section{Valores erróneos}
En lo referente a los valores erróneos se trata de una de las tareas más arduas 
del proyecto, puesto que parte de este se desarrolló de forma visual diréctamente
sobre el conjunto de datos de los diferentes sensores y el pluviómetro, siendo, 
de esta manera, en su mayoría un proceso manual.

Para tal propósito se estableció una columna adicional al conjunto de datos original
que indicaba la validez de la muestra concreta.

% Sensor 1
\imagen{processed_data/sensor1.png}{Datos procesados: sensor 1}{1}

% Sensor 2
\imagen{processed_data/sensor2.png}{Datos procesados: sensor 2}{1}

% Sensor 3
\imagen{processed_data/sensor3.png}{Datos procesados: sensor 3}{1}

% Sensor 4
\imagen{processed_data/sensor4.png}{Datos procesados: sensor 4}{1}

% Sensor 5
\imagen{processed_data/sensor5.png}{Datos procesados: sensor 5}{1}

% Sensor 6
\imagen{processed_data/sensor6.png}{Datos procesados: sensor 6}{1}

% Sensor 7
\imagen{processed_data/sensor7.png}{Datos procesados: sensor 7}{1}

% Sensor 8
\imagen{processed_data/sensor8.png}{Datos procesados: sensor 8}{1}

Hablándo en términos de sesgos, es posible que al realizar la recuperación de estos
valores en los sensores mencionados, puedan haberse sesgado en cierta medida
los conjuntos de datos, sin embargo, los datos no dejarán de estar más cercanos a los
valores reales y, por tanto, contener un error más reducido que las muestras originales
que contenían esta casuística.