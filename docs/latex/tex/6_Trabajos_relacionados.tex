\capitulo{6}{Trabajos relacionados}

En esta sección se introducirán los trabajos que guardan relación con el proyecto desarrollado,
es decir, aquellos cuyo objetivos, técnicas y modelos sean similares.

Y es que, como se ha comentado en la introducción, la optimización de recursos hídricos en los 
últimos años ha crecido en importancia debido principalmente a la escasez de precipitaciones
por el cambio climático, lo que provoca que las horas y cantidad de riego disminuyan en todo el territorio
español, especialmente en las zonas del sur de la Península Ibérica \cite{misc:PNR,misc:PNRdemanda}.

\section{\textit{Weather forecasting model using Artificial Neural Network}}
En este artículo publicado en 2012 se presenta un método de predicción del tiempo atmosférico
empleando modelos neuronales \textit{feed-forward} con retropropagación, de forma que se busca
explorar la importancia de las redes neuronales artificiales y sus componentes unitarios.

Se experimenta, entonces, cómo incrementa el rendimiento al aumentar el número de neuronas,
el número de capas ocultas del modelo y se presentan las conclusiones de integrar ambas decisiones
del diseño. En esta ocasión se emplean en los datos variables de temperatura (máximos y 
mínimos), así como velocidades diarias del viento \cite{ABHISHEK2012311}.

\section{\textit{Humidity forecasting in a potato plantation using time-series neural models}}
En el año 2022 el Grupo de Inteligencia Computacional Aplicada (GICAP) y el Composting Research Group
(UBUCOMP) de la Universidad de Burgos publican un artículo dónde presentan un mecanismo de control
de irrigación de una plantación de patatas en la localidad de Cabia en la provincia de Burgos.

En este trabajo se muestrean diversas variables del suelo como la temperatura del aire,
las precipitaciones, la altura y anchura de las plantas, el porcentaje de humedad y su contenido
en nitrógeno entre otros valores, para, de este modo, predecir la necesidad de realizar o no la
irrigación de la plantación y optimizar los recursos hídricos.

En este artículo se propone emplear diferentes tipos de interpolación como técnica básica para
comparar resultados y varios modelos neuronales como métodos de refinado de predicciones de las series de tiempo. 
Se utilizarán, de esta forma, los modelos \textit{Non-linear Input-Output} (\textit{NIO}), 
\textit{Non-linear Autoregressive with Exogenous Input} (\textit{NARX}) y \textit{Non-linear Autoregressive} (\textit{NAR}) \cite{article:baseca2022}.

\section{\textit{BMAE-Net: A Data-Driven Weather Prediction Network for Smart Agriculture}}
En este artículo publicado en 2023 se presenta un método de predicción de cambios de series de tiempo
meteorológicas basadas en \textit{encoders} y \textit{decoders} de atención multicabezal
\cite{article:kong2023} optimizados mediante estrategias de inferencia bayesiana. 

Este mecanismo permite diseñar una novedosa estructura neuronal que mejora el 
rendimiento de compatibilidad para predicciones de distinta duración \cite{article:kong2023}.  

Se emplean, de esta manera, diferentes componentes; uno de ellos es un módulo bayesiano-GRU, utilizado
para muestrear los pesos de la red con una distribución de probabilidad y optimizar sus parámetros,
en lugar de establecer un peso concreto en la modelo neuronal.

Además, cuenta con un módulo de atención multicabeza bayesiano, que se encarga de la predicción de 
los diferentes intervalos y mejora la capacidad de extracción de características. 
